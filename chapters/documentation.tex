\chapter{Documentation}

\thesislettrine{T}{his thesis class} was developed for my masters thesis, and I
have now made it public so that others may use.  The class itself is based on
|scrreprt| from the \href{https://www.ctan.org/pkg/koma-script}{KOMA-Script
  bundle}, and if compiled with \LuaLaTeX{} this thesis template uses the free
\href{http://www.georgduffner.at/ebgaramond/}{\emph{EB Garamond}} font by Georg
Duffner.

Since multiple compilation runs are usually required in order to get all the
cross-references correct (and then there's |biber| for the bibliography, maybe
|makeglossaries| if you have a glosary, etc.), I recommend using a compilation
tool such as \href{https://www.ctan.org/pkg/latexmk}{|latexmk|},
\href{https://launchpad.net/rubber/}{Rubber} or
\href{https://www.ctan.org/pkg/arara}{arara}.  Included with this template is a
|.latexmkrc| configuration which uses \LuaLaTeX{} by default and is capable of
handling externalized Ti\emph{k}Z images.  To compile the thesis with |latexmk|,
it is just a matter of typing |latexmk thesis.tex| in the command line (or
setting up your favourite \textsc{ide} to do this for you).

Since a thesis will likely become quite a big document, it is recommended to
split it across multiple files and then using |\include| to combine them in the
main document.  The advantaged of using |\include| over |\input| is that each
included file has its own |.aux| file associated with it (which stores
information about sections, labels, links, \dots).  It is then possible to use
|\includeonly| in the preamble so that only certain files are included, but this
still allows cross-references to work properly.

\begin{codeexample}[execute code=false]
\documentclass{thesis}
\includeonly{
  chapters/introduction,
  chapters/documentation
}
\begin{document}
\include{chapters/introduction}
\chapter{Documentation}

\thesislettrine{T}{his thesis class} was developed for my masters thesis, and I
have now made it public so that others may use.  The class itself is based on
|scrreprt| from the \href{https://www.ctan.org/pkg/koma-script}{KOMA-Script
  bundle}, and if compiled with \LuaLaTeX{} this thesis template uses the free
\href{http://www.georgduffner.at/ebgaramond/}{\emph{EB Garamond}} font by Georg
Duffner.

Since multiple compilation runs are usually required in order to get all the
cross-references correct (and then there's |biber| for the bibliography, maybe
|makeglossaries| if you have a glosary, etc.), I recommend using a compilation
tool such as \href{https://www.ctan.org/pkg/latexmk}{|latexmk|},
\href{https://launchpad.net/rubber/}{Rubber} or
\href{https://www.ctan.org/pkg/arara}{arara}.  Included with this template is a
|.latexmkrc| configuration which uses \LuaLaTeX{} by default and is capable of
handling externalized Ti\emph{k}Z images.  To compile the thesis with |latexmk|,
it is just a matter of typing |latexmk thesis.tex| in the command line (or
setting up your favourite \textsc{ide} to do this for you).

Since a thesis will likely become quite a big document, it is recommended to
split it across multiple files and then using |\include| to combine them in the
main document.  The advantaged of using |\include| over |\input| is that each
included file has its own |.aux| file associated with it (which stores
information about sections, labels, links, \dots).  It is then possible to use
|\includeonly| in the preamble so that only certain files are included, but this
still allows cross-references to work properly.

\begin{codeexample}[execute code=false]
\documentclass{thesis}
\includeonly{
  chapters/introduction,
  chapters/documentation
}
\begin{document}
\include{chapters/introduction}
\chapter{Documentation}

\thesislettrine{T}{his thesis class} was developed for my masters thesis, and I
have now made it public so that others may use.  The class itself is based on
|scrreprt| from the \href{https://www.ctan.org/pkg/koma-script}{KOMA-Script
  bundle}, and if compiled with \LuaLaTeX{} this thesis template uses the free
\href{http://www.georgduffner.at/ebgaramond/}{\emph{EB Garamond}} font by Georg
Duffner.

Since multiple compilation runs are usually required in order to get all the
cross-references correct (and then there's |biber| for the bibliography, maybe
|makeglossaries| if you have a glosary, etc.), I recommend using a compilation
tool such as \href{https://www.ctan.org/pkg/latexmk}{|latexmk|},
\href{https://launchpad.net/rubber/}{Rubber} or
\href{https://www.ctan.org/pkg/arara}{arara}.  Included with this template is a
|.latexmkrc| configuration which uses \LuaLaTeX{} by default and is capable of
handling externalized Ti\emph{k}Z images.  To compile the thesis with |latexmk|,
it is just a matter of typing |latexmk thesis.tex| in the command line (or
setting up your favourite \textsc{ide} to do this for you).

Since a thesis will likely become quite a big document, it is recommended to
split it across multiple files and then using |\include| to combine them in the
main document.  The advantaged of using |\include| over |\input| is that each
included file has its own |.aux| file associated with it (which stores
information about sections, labels, links, \dots).  It is then possible to use
|\includeonly| in the preamble so that only certain files are included, but this
still allows cross-references to work properly.

\begin{codeexample}[execute code=false]
\documentclass{thesis}
\includeonly{
  chapters/introduction,
  chapters/documentation
}
\begin{document}
\include{chapters/introduction}
\chapter{Documentation}

\thesislettrine{T}{his thesis class} was developed for my masters thesis, and I
have now made it public so that others may use.  The class itself is based on
|scrreprt| from the \href{https://www.ctan.org/pkg/koma-script}{KOMA-Script
  bundle}, and if compiled with \LuaLaTeX{} this thesis template uses the free
\href{http://www.georgduffner.at/ebgaramond/}{\emph{EB Garamond}} font by Georg
Duffner.

Since multiple compilation runs are usually required in order to get all the
cross-references correct (and then there's |biber| for the bibliography, maybe
|makeglossaries| if you have a glosary, etc.), I recommend using a compilation
tool such as \href{https://www.ctan.org/pkg/latexmk}{|latexmk|},
\href{https://launchpad.net/rubber/}{Rubber} or
\href{https://www.ctan.org/pkg/arara}{arara}.  Included with this template is a
|.latexmkrc| configuration which uses \LuaLaTeX{} by default and is capable of
handling externalized Ti\emph{k}Z images.  To compile the thesis with |latexmk|,
it is just a matter of typing |latexmk thesis.tex| in the command line (or
setting up your favourite \textsc{ide} to do this for you).

Since a thesis will likely become quite a big document, it is recommended to
split it across multiple files and then using |\include| to combine them in the
main document.  The advantaged of using |\include| over |\input| is that each
included file has its own |.aux| file associated with it (which stores
information about sections, labels, links, \dots).  It is then possible to use
|\includeonly| in the preamble so that only certain files are included, but this
still allows cross-references to work properly.

\begin{codeexample}[execute code=false]
\documentclass{thesis}
\includeonly{
  chapters/introduction,
  chapters/documentation
}
\begin{document}
\include{chapters/introduction}
\include{chapters/documentation}
\include{chapter/lorem}
\include{chapter/ipsum}
\include{chapter/conclusion}
\end{document}
\end{codeexample}


\section{Customizations}
\label{sec:customizations}

This thesis class makes use `keys' and `styles' in the same as a lot of
Ti\emph{k}Z packages.  This allows for many customizations to be done easily and
consistently.  The various keys are defined below along with descriptions of
what each does.  They can be customized using the |\thesisset| command.

\begin{command}{\thesisset\marg{keys}}
  This facilitates the change of thesis keys and styles and by default will
  adjust keys in the |/thesis| family of keys.

\begin{codeexample}[execute code=false]
\thesisset{
  title/logo={
    \includegraphics[width=6cm]{my-university-logo.png}
  },
  title/logo skip=3cm,
}
\end{codeexample}
\end{command}

There are a few cases where customizations are handled by other packages, or
where commands are more appropriate.

\subsection{Font Handling}
\label{subsec:font_handling}

The thesis template makes use of a few font, especially in the title, and in
order to handle all these fonts and switch between them easily, a convenience
functions are introduced.

At the core of all this is the |font| handler.  A key can be designated as a
font with |.is font| and can be subsequently accessed with |\thesisgetfont| or
|\thesisfont|.

\begin{handler}{.is font}
  This defines a particular key to be an actual font.  This creates a new family
  with the given key, and defines the following subkeys with default values.
  The font can then be used in conjunction with |\thesisgetfont| and
  |\thesisfont|.

  Since the handler simply turns the key into a family of keys, it is possible
  to add additional subkeys without affecting the functionality of the font
  key.  For example:

  \begin{codeexample}[]
  \thesisset{
    cormorant/.is font,
    cormorant/font=\fontspec{Cormorant},
    cormorant/color=red!75!blue,
    cormorant/shape=\scshape,
    % The following is NOT part of the font keys
    cormorant/text/.initial={I'm in Cormorant!}
  }
  \thesisgetfont{cormorant}{\thesisget{cormorant/text}}
  \end{codeexample}

  The subkeys defined, along with their respective default values, are:
  \begin{description}[font=\normalfont]
  \item[\marg{key}|/size|] (default |1em|) \newline
    Stores the size (with units) of the font;
  \item[\marg{key}|/font|] (default |\normalfont|) \newline
    Defines the font.  Can be used in conjunction with |fontspec|'s |\fontspec|
    command;
  \item[\marg{key}|/color|] (default |black|) \newline
    Changes the colour;
  \item[\marg{key}|/shape|] (default |\upshape|) \newline
    Adjusts the font shape (i.e. |\itshape|, |\scshape|);
  \item[\marg{key}|/series|] (default |\mdseries|) \newline
    Changes the font series (i.e. |\bfseries|, |\mdseries|).
  \end{description}
\end{handler}

\begin{command}{\thesisgetfont\marg{font key}}
  This fetches a font key (that is, one defined with |.is font|), and applies
  the corresponding font to the current context.  This command is analogous to
  commands such as |\bfseries|, |\scshape|, \dots{} in that they will apply to
  the \emph{entire} context and thus should be used within an environment and/or
  braces |{}| in order to restrict the extent of the command.

\begin{codeexample}[]
\thesisset{
  cormorant/.is font,
  cormorant/font=\fontspec{Cormorant},
  cormorant/color=red!75!blue,
  cormorant/shape=\scshape,
}
{\thesisgetfont{cormorant} Hello, World!} Goodbye
\end{codeexample}
\end{command}

\begin{command}{\thesisfont\marg{font key}\marg{text}}
  This fetches a font key (that is, one defined with |.is font|), and applies
  the corresponding font to the specified text.  This command is analogous to
  commands such as |\textbf|\marg{text}, |\textsc|\marg{text}, \dots{} in that
  they apply only the the specified text.

\begin{codeexample}[]
\thesisset{
  cormorant/.is font,
  cormorant/font=\fontspec{Cormorant},
  cormorant/color=red!25!blue,
  cormorant/shape=\scshape,
}
\thesisfont{cormorant}{Hello, World!} Goodbye
\end{codeexample}
\end{command}


\section{Document Information}
\label{sec:document_information}

Information about authors, supervisors, title, \dots{} should be set in the
preamble using the commands provided below or by setting the appropriate key.
If needed, the values can subsequently be accessed from the key.  In order to be
compatible with the usual way \LaTeX{} handles this information, the associated
|\@...| command is also defined

\begin{keylist}{
    /thesis/author/first name,
    /thesis/author/last name}
  Stores the author's first and last name.

  \begin{command}{\author\marg{first name}\marg{last name}}
    A convenience function to set the author.
  \end{command}
\end{keylist}

\begin{key}{/thesis/author/orcid}
  Stores the author's \textsc{Orcid}.

  \begin{command}{\orcid\marg{orcid}}
    A convenience function to set the \textsc{Orcid}.
  \end{command}
\end{key}

\begin{keylist}{
    /thesis/supervisors/\meta{count}/first name,
    /thesis/supervisors/\meta{count}/last name}
  Stores the supervisor's first and last name, where \meta{count} is an integer
  (starting from 1) allowing for multiple supervisors to be handled.  The
  information should always be set with the |\supervisor| command.

  Internally, the number of supervisors in total is recorded in the counter
  |supervisorcount|.  It is then possible to programmatically access each
  supervisor with:

\begin{codeexample}[]
\foreach \i in {1, ..., \arabic{supervisorcount}} {
  \thesisget{supervisors/\i/first name}
  ~\thesisget{supervisors/\i/last name}
}
\end{codeexample}

  \begin{command}{\supervisor\marg{first name}\marg{last name}}
    A convenience function to input supervisors.  This automatically handles the
    mechanics to keep track of the number of supervisors and is the preferred
    way to set the information.  The order in which the supervisors are
    specified is kept.
  \end{command}
\end{keylist}

\begin{key}{/thesis/logo}
  Stores the code for the logo.  Typically, this will be |\includegraphics{...}|.

  \begin{command}{\logo\marg{code}}
    A convenience function to set the logo.
  \end{command}
\end{key}

\begin{key}{/thesis/degree title}
  Stores the degree title.

  \begin{command}{\degreetitle\marg{text}}
    A convenience function to set the degree title
  \end{command}
\end{key}

\begin{key}{/thesis/university}
  Stores the university name.

  \begin{command}{\university\marg{text}}
    A convenience function to set the university name
  \end{command}
\end{key}

\begin{key}{/thesis/department}
  Stores the university department.

  \begin{command}{\department\marg{text}}
    A convenience function to set the university department.
  \end{command}
\end{key}

\begin{key}{/thesis/subject}
  Stores the subject.  This is used primarily to set the |subject| property in
  the \textsc{pdf}.

  \begin{command}{\subject\marg{text}}
    A convenience function to set the subject.
  \end{command}
\end{key}

\begin{key}{/thesis/keywords}
  Stores a list of keywords relating to the thesis.  This is used primarily to
  set the |keywords| property in the \textsc{pdf}.

  \begin{command}{\keywords\marg{text}}
    A convenience function to set the keywords.
  \end{command}
\end{key}


\section{Formatting}
\label{sec:formatting}

The |thesis| class is based on the |scrreprt| class from the
\href{https://www.ctan.org/pkg/koma-script}{KOMA-Script bundle}.  The |thesis|
customizes a lot of the formatting from the |scrreprt|, but these can always be
adjusted further in the preamble of the thesis.  In particular, the font and
formatting of section titles can be adjust with the |\setkomafont| command.

The formatting of the title can be extensively adjusted using keys provided by
the |thesis| class, as documented in \cref{subsec:title_page}.  There are other
miscellaneous features provided by the font, such as lettrines in
\cref{subsec:lettrines}.

\subsection{Draft Mode}
\label{subsec:draft_mode}

Since a thesis can take a long time to compile, it can be worthwhile to enable
|draft| mode by adding |draft| as an option to the document class:

\begin{codeexample}[execute code=false]
\documentclass[draft]{thesis}
\end{codeexample}

When draft mode is enabled, the |thesis| class will automatically add line
numbers and highlight certain errors with a black box near the error.  More
importantly, certain features which are known to significantly increase
compilation time are disabled.  Some changes include:
\begin{itemize}
\item Most features by \href{http://ctan.org/pkg/microtype}{|microtype|} are
  disabled;
\item Hyperlinks are not generated by
  \href{http://ctan.org/pkg/hyperref}{|hyperref|};
\item Images are replaced by placeholders.
\end{itemize}

When the final version of the document is to be generated, |draft| should be
replaced with |final|:

\begin{codeexample}[execute code=false]
\documentclass[final]{thesis}
\end{codeexample}

This ensures that all the features are enabled.

\subsection{Page Layout}
\label{subsec:page_layout}

The page layout can be adjusted using the |\geometry| command from the extensive
\href{http://ctan.org/pkg/geometry}{|geometry|} package.  Please refer to the
|geometry| documentation for a full list of all of its capabilities.  As a rule
of thumb (and ultimately, it is ultimately a matter of taste), it is advisable
to set the length of each line so that there are no more than 60--90 characters
as the line length is long enough to read fairly fast without being so long that
the eye can't find the start of the next line.  The text width can be adjusted
|textwidth| key which will adjust the margins automatically to fit.

\begin{codeexample}[execute code=false]
\geometry{
  textwidth=13cm,
  vmargin={3cm,3cm},
}
\end{codeexample}

\subsection{Title Page}
\label{subsec:title_page}

The thesis title is generated using the |\maketitle| command and can be
extensively customized using various keys provided by the |thesis| class, as
described below.  The information for the title is obtained from the document
information (see \cref{sec:document_information}).  All keys to customize the
title page fall under |/thesis/title|, and since there are many components, they
are further subcategorized as seen below.

\subsubsection{Logo}
\label{subsubsec:logo}

\begin{key}{/thesis/title/logo/skip (initially 2em)}
  The vertical skip after the logo when the logo is present.
\end{key}

\subsubsection{University Name and Department}
\label{subsubsec:university_logo_name_and_department}

\begin{fontkeylist}{/thesis/title/university/name, /thesis/title/university/department}
  The font for the university name and department respectively.
\end{fontkeylist}

\begin{keylist}{
    /thesis/title/university/name/skip (initially 1em),
    /thesis/title/university/department/skip (initially 5em)}
  The vertical skip after the university name and department respectively.  They
  are only used if corresponding text is not empty.
\end{keylist}

\subsubsection{Title and Subtitle}
\label{subsubsec:title_and_subtitle}

\begin{fontkeylist}{
    /thesis/title/title,
    /thesis/title/subtitle}
  The font for the thesis title and subtitle.
\end{fontkeylist}

\begin{key}{/thesis/title/title/skip (initially 5em)}
  The vertical skip \emph{after both} title and subtitle.
\end{key}

\begin{key}{/thesis/title/subtitle/skip (initially 0.5em)}
  The vertical skip \emph{between} the title and subtitle.
\end{key}

\subsubsection{Author}
\label{subsubsec:author}

\begin{fontkeylist}{
    /thesis/title/author/description,
    /thesis/title/author/first name,
    /thesis/title/author/last name,
    /thesis/title/author/orcid}
  Specifies the font for the author description, the author's first and last
  name, and the \textsc{orcid}.
\end{fontkeylist}

\begin{key}{/thesis/title/author/description/text (initially By)}
  The text introducing the author.
\end{key}

\begin{key}{/thesis/title/author/separator (initially \string\space)}
  The separator between the author's first and last name.
\end{key}

\begin{keylist}{
    /thesis/title/author/description/skip (initially 0.2em),
    /thesis/title/author/orcid/skip (initially -0.2em),
    /thesis/title/author/skip (initially 3em)}
  The vertical skip between the description text and the author's name, between
  the author's name and the \textsc{orcid}, and after all the author's details
  respectively.
\end{keylist}

\subsubsection{Supervisor}
\label{subsubsec:supervisor}

\begin{fontkeylist}{
    /thesis/supervisor/description,
    /thesis/supervisor/first name,
    /thesis/supervisor/last name}
  Specifies the font for the supervisor description, and the supervisor(s)'s
  first and last name.
\end{fontkeylist}

\begin{key}{/thesis/supervisor/description/text (initially Under the supervision of)}
  The text introducing the supervisor(s).
\end{key}

\begin{key}{/thesis/supervisor/separator (initially \string\space)}
  The separator between each supervisor's first and last name.
\end{key}

\begin{keylist}{
    /thesis/supervisor/description/skip (initially 0.2em),
    /thesis/supervisor/inter skip (initially 0pt),
    /thesis/supervisor/skip (initially 3em)}
  The vertical skip between the description and the first supervisor's name,
  between each supervisor, and between the last supervisor and after the last supervisor.
\end{keylist}

\subsubsection{Degree}
\label{subsubsec:degree}

\begin{fontkeylist}{
    /thesis/degree/description,
    /thesis/degree/name}
  Specifies the font for the degree description and the degree name.
\end{fontkeylist}

\begin{key}{/thesis/degree/description/text (initially Submitted in the fulfillment of)}
  The text introducing the degree.
\end{key}

\begin{keylist}{
    /thesis/degree/description/skip (initially 0.2em),
    /thesis/degree/skip (initially 0em plus 1 fill)}
  The vertical skip between the degree description
\end{keylist}

\subsubsection{Date}
\label{subsubsec:date}

\begin{fontkey}{/thesis/title/date}
  Adjust the font for the date.
\end{fontkey}

\subsection{Lettrines}
\label{subsec:lettrines}

\begin{command}{\thesislettrine\marg{letter}\marg{text}}
  Inserts a \href{https://en.wikipedia.org/wiki/Initial}{lettrine} (or
  \emph{initial} or \emph{drop cap}) at the start of a paragraph, and formats
  the remaining text specified in \marg{text} using small caps.  Conventionally,
  it is either the rest of the first word or noun phrase that is included the
  \marg{text}.

  \hfill\begin{minipage}{0.9\textwidth}
    % For some reason, the lettrine doesn't display nicely from the codeexample.
    \thesislettrine{A}{simple example of a lettrine} in use.  \lipsum[2]
  \end{minipage}

\begin{codeexample}[execute code=false]
\thesislettrine{A}{simple example of a lettrine} in use.  \lipsum[2]
\end{codeexample}

  If \LuaLaTeX{} is available, the lettrine will make use of the EB Garamond
  lettrine glyphs.  EB Garamond provides two separate fonts for the foreground
  and background of the lettrines; alternative fonts can be used by adjusting
  |\lettrinebg| and |\lettrinefg| (see below).  If \LuaLaTeX{} is not available,
  the |\thesislettrine| will simply insert a large first character and use the
  primary colour.

  \hfill\begin{minipage}{0.9\textwidth}
    % For some reason, the lettrine doesn't display nicely from the codeexample.
    \setcounter{DefaultLines}{3}
    \renewcommand{\DefaultLoversize}{0}
    \renewcommand{\DefaultLraise}{0}
    \setlength{\DefaultNindent}{0.5ex}
    \lettrine{\textcolor{primary}{A}}{simple example of a lettrine} in use.  \lipsum[2]
  \end{minipage}

\begin{codeexample}[execute code=false]
\thesislettrine{A}{simple example of a lettrine} in use.  \lipsum[2]
\end{codeexample}

  Internally, |\thesislettrine| makes use of the |\lettrine| command from the
  \href{http://ctan.org/pkg/lettrine}{|lettrine|} package which provides various
  other commands to customize the appearance of the lettrines.  In using
  \LuaLaTeX{}, the foreground and background fonts can be adjusted using the
  following two commands:

\end{command}

% Local Variables:
% TeX-master: "../thesis.tex"
% End:
\chapter{Lorem}

\lipsum[1-3]

\section{Quiset Habitant}

\lipsum[4-10]

\subsection{Nullam Cursus}

\lipsum[11-13]

\subsection{Donec et Mi}

\lipsum[14-17]

\section{Dignissim Rutrum}

\lipsum[20-26]

\subsection{Fringilla}

\lipsum[26-28]

\subsection{Aliquet}

\lipsum[29-30]

% Local Variables:
% TeX-master: "../thesis.tex"
% End:
\chapter{Ipsum}

\lipsum[1-3]

\section{Quiset Habitant}

\lipsum[4-10]

\subsection{Nullam Cursus}

\lipsum[11-13]

\subsection{Donec et Mi}

\lipsum[14-17]

\section{Dignissim Rutrum}

\lipsum[20-26]

\subsection{Fringilla}

\lipsum[26-28]

\subsection{Aliquet}

\lipsum[29-30]

% Local Variables:
% TeX-master: "../thesis.tex"
% End:
\include{chapter/conclusion}
\end{document}
\end{codeexample}


\section{Customizations}
\label{sec:customizations}

This thesis class makes use `keys' and `styles' in the same as a lot of
Ti\emph{k}Z packages.  This allows for many customizations to be done easily and
consistently.  The various keys are defined below along with descriptions of
what each does.  They can be customized using the |\thesisset| command.

\begin{command}{\thesisset\marg{keys}}
  This facilitates the change of thesis keys and styles and by default will
  adjust keys in the |/thesis| family of keys.

\begin{codeexample}[execute code=false]
\thesisset{
  title/logo={
    \includegraphics[width=6cm]{my-university-logo.png}
  },
  title/logo skip=3cm,
}
\end{codeexample}
\end{command}

There are a few cases where customizations are handled by other packages, or
where commands are more appropriate.

\subsection{Font Handling}
\label{subsec:font_handling}

The thesis template makes use of a few font, especially in the title, and in
order to handle all these fonts and switch between them easily, a convenience
functions are introduced.

At the core of all this is the |font| handler.  A key can be designated as a
font with |.is font| and can be subsequently accessed with |\thesisgetfont| or
|\thesisfont|.

\begin{handler}{.is font}
  This defines a particular key to be an actual font.  This creates a new family
  with the given key, and defines the following subkeys with default values.
  The font can then be used in conjunction with |\thesisgetfont| and
  |\thesisfont|.

  Since the handler simply turns the key into a family of keys, it is possible
  to add additional subkeys without affecting the functionality of the font
  key.  For example:

  \begin{codeexample}[]
  \thesisset{
    cormorant/.is font,
    cormorant/font=\fontspec{Cormorant},
    cormorant/color=red!75!blue,
    cormorant/shape=\scshape,
    % The following is NOT part of the font keys
    cormorant/text/.initial={I'm in Cormorant!}
  }
  \thesisgetfont{cormorant}{\thesisget{cormorant/text}}
  \end{codeexample}

  The subkeys defined, along with their respective default values, are:
  \begin{description}[font=\normalfont]
  \item[\marg{key}|/size|] (default |1em|) \newline
    Stores the size (with units) of the font;
  \item[\marg{key}|/font|] (default |\normalfont|) \newline
    Defines the font.  Can be used in conjunction with |fontspec|'s |\fontspec|
    command;
  \item[\marg{key}|/color|] (default |black|) \newline
    Changes the colour;
  \item[\marg{key}|/shape|] (default |\upshape|) \newline
    Adjusts the font shape (i.e. |\itshape|, |\scshape|);
  \item[\marg{key}|/series|] (default |\mdseries|) \newline
    Changes the font series (i.e. |\bfseries|, |\mdseries|).
  \end{description}
\end{handler}

\begin{command}{\thesisgetfont\marg{font key}}
  This fetches a font key (that is, one defined with |.is font|), and applies
  the corresponding font to the current context.  This command is analogous to
  commands such as |\bfseries|, |\scshape|, \dots{} in that they will apply to
  the \emph{entire} context and thus should be used within an environment and/or
  braces |{}| in order to restrict the extent of the command.

\begin{codeexample}[]
\thesisset{
  cormorant/.is font,
  cormorant/font=\fontspec{Cormorant},
  cormorant/color=red!75!blue,
  cormorant/shape=\scshape,
}
{\thesisgetfont{cormorant} Hello, World!} Goodbye
\end{codeexample}
\end{command}

\begin{command}{\thesisfont\marg{font key}\marg{text}}
  This fetches a font key (that is, one defined with |.is font|), and applies
  the corresponding font to the specified text.  This command is analogous to
  commands such as |\textbf|\marg{text}, |\textsc|\marg{text}, \dots{} in that
  they apply only the the specified text.

\begin{codeexample}[]
\thesisset{
  cormorant/.is font,
  cormorant/font=\fontspec{Cormorant},
  cormorant/color=red!25!blue,
  cormorant/shape=\scshape,
}
\thesisfont{cormorant}{Hello, World!} Goodbye
\end{codeexample}
\end{command}


\section{Document Information}
\label{sec:document_information}

Information about authors, supervisors, title, \dots{} should be set in the
preamble using the commands provided below or by setting the appropriate key.
If needed, the values can subsequently be accessed from the key.  In order to be
compatible with the usual way \LaTeX{} handles this information, the associated
|\@...| command is also defined

\begin{keylist}{
    /thesis/author/first name,
    /thesis/author/last name}
  Stores the author's first and last name.

  \begin{command}{\author\marg{first name}\marg{last name}}
    A convenience function to set the author.
  \end{command}
\end{keylist}

\begin{key}{/thesis/author/orcid}
  Stores the author's \textsc{Orcid}.

  \begin{command}{\orcid\marg{orcid}}
    A convenience function to set the \textsc{Orcid}.
  \end{command}
\end{key}

\begin{keylist}{
    /thesis/supervisors/\meta{count}/first name,
    /thesis/supervisors/\meta{count}/last name}
  Stores the supervisor's first and last name, where \meta{count} is an integer
  (starting from 1) allowing for multiple supervisors to be handled.  The
  information should always be set with the |\supervisor| command.

  Internally, the number of supervisors in total is recorded in the counter
  |supervisorcount|.  It is then possible to programmatically access each
  supervisor with:

\begin{codeexample}[]
\foreach \i in {1, ..., \arabic{supervisorcount}} {
  \thesisget{supervisors/\i/first name}
  ~\thesisget{supervisors/\i/last name}
}
\end{codeexample}

  \begin{command}{\supervisor\marg{first name}\marg{last name}}
    A convenience function to input supervisors.  This automatically handles the
    mechanics to keep track of the number of supervisors and is the preferred
    way to set the information.  The order in which the supervisors are
    specified is kept.
  \end{command}
\end{keylist}

\begin{key}{/thesis/logo}
  Stores the code for the logo.  Typically, this will be |\includegraphics{...}|.

  \begin{command}{\logo\marg{code}}
    A convenience function to set the logo.
  \end{command}
\end{key}

\begin{key}{/thesis/degree title}
  Stores the degree title.

  \begin{command}{\degreetitle\marg{text}}
    A convenience function to set the degree title
  \end{command}
\end{key}

\begin{key}{/thesis/university}
  Stores the university name.

  \begin{command}{\university\marg{text}}
    A convenience function to set the university name
  \end{command}
\end{key}

\begin{key}{/thesis/department}
  Stores the university department.

  \begin{command}{\department\marg{text}}
    A convenience function to set the university department.
  \end{command}
\end{key}

\begin{key}{/thesis/subject}
  Stores the subject.  This is used primarily to set the |subject| property in
  the \textsc{pdf}.

  \begin{command}{\subject\marg{text}}
    A convenience function to set the subject.
  \end{command}
\end{key}

\begin{key}{/thesis/keywords}
  Stores a list of keywords relating to the thesis.  This is used primarily to
  set the |keywords| property in the \textsc{pdf}.

  \begin{command}{\keywords\marg{text}}
    A convenience function to set the keywords.
  \end{command}
\end{key}


\section{Formatting}
\label{sec:formatting}

The |thesis| class is based on the |scrreprt| class from the
\href{https://www.ctan.org/pkg/koma-script}{KOMA-Script bundle}.  The |thesis|
customizes a lot of the formatting from the |scrreprt|, but these can always be
adjusted further in the preamble of the thesis.  In particular, the font and
formatting of section titles can be adjust with the |\setkomafont| command.

The formatting of the title can be extensively adjusted using keys provided by
the |thesis| class, as documented in \cref{subsec:title_page}.  There are other
miscellaneous features provided by the font, such as lettrines in
\cref{subsec:lettrines}.

\subsection{Draft Mode}
\label{subsec:draft_mode}

Since a thesis can take a long time to compile, it can be worthwhile to enable
|draft| mode by adding |draft| as an option to the document class:

\begin{codeexample}[execute code=false]
\documentclass[draft]{thesis}
\end{codeexample}

When draft mode is enabled, the |thesis| class will automatically add line
numbers and highlight certain errors with a black box near the error.  More
importantly, certain features which are known to significantly increase
compilation time are disabled.  Some changes include:
\begin{itemize}
\item Most features by \href{http://ctan.org/pkg/microtype}{|microtype|} are
  disabled;
\item Hyperlinks are not generated by
  \href{http://ctan.org/pkg/hyperref}{|hyperref|};
\item Images are replaced by placeholders.
\end{itemize}

When the final version of the document is to be generated, |draft| should be
replaced with |final|:

\begin{codeexample}[execute code=false]
\documentclass[final]{thesis}
\end{codeexample}

This ensures that all the features are enabled.

\subsection{Page Layout}
\label{subsec:page_layout}

The page layout can be adjusted using the |\geometry| command from the extensive
\href{http://ctan.org/pkg/geometry}{|geometry|} package.  Please refer to the
|geometry| documentation for a full list of all of its capabilities.  As a rule
of thumb (and ultimately, it is ultimately a matter of taste), it is advisable
to set the length of each line so that there are no more than 60--90 characters
as the line length is long enough to read fairly fast without being so long that
the eye can't find the start of the next line.  The text width can be adjusted
|textwidth| key which will adjust the margins automatically to fit.

\begin{codeexample}[execute code=false]
\geometry{
  textwidth=13cm,
  vmargin={3cm,3cm},
}
\end{codeexample}

\subsection{Title Page}
\label{subsec:title_page}

The thesis title is generated using the |\maketitle| command and can be
extensively customized using various keys provided by the |thesis| class, as
described below.  The information for the title is obtained from the document
information (see \cref{sec:document_information}).  All keys to customize the
title page fall under |/thesis/title|, and since there are many components, they
are further subcategorized as seen below.

\subsubsection{Logo}
\label{subsubsec:logo}

\begin{key}{/thesis/title/logo/skip (initially 2em)}
  The vertical skip after the logo when the logo is present.
\end{key}

\subsubsection{University Name and Department}
\label{subsubsec:university_logo_name_and_department}

\begin{fontkeylist}{/thesis/title/university/name, /thesis/title/university/department}
  The font for the university name and department respectively.
\end{fontkeylist}

\begin{keylist}{
    /thesis/title/university/name/skip (initially 1em),
    /thesis/title/university/department/skip (initially 5em)}
  The vertical skip after the university name and department respectively.  They
  are only used if corresponding text is not empty.
\end{keylist}

\subsubsection{Title and Subtitle}
\label{subsubsec:title_and_subtitle}

\begin{fontkeylist}{
    /thesis/title/title,
    /thesis/title/subtitle}
  The font for the thesis title and subtitle.
\end{fontkeylist}

\begin{key}{/thesis/title/title/skip (initially 5em)}
  The vertical skip \emph{after both} title and subtitle.
\end{key}

\begin{key}{/thesis/title/subtitle/skip (initially 0.5em)}
  The vertical skip \emph{between} the title and subtitle.
\end{key}

\subsubsection{Author}
\label{subsubsec:author}

\begin{fontkeylist}{
    /thesis/title/author/description,
    /thesis/title/author/first name,
    /thesis/title/author/last name,
    /thesis/title/author/orcid}
  Specifies the font for the author description, the author's first and last
  name, and the \textsc{orcid}.
\end{fontkeylist}

\begin{key}{/thesis/title/author/description/text (initially By)}
  The text introducing the author.
\end{key}

\begin{key}{/thesis/title/author/separator (initially \string\space)}
  The separator between the author's first and last name.
\end{key}

\begin{keylist}{
    /thesis/title/author/description/skip (initially 0.2em),
    /thesis/title/author/orcid/skip (initially -0.2em),
    /thesis/title/author/skip (initially 3em)}
  The vertical skip between the description text and the author's name, between
  the author's name and the \textsc{orcid}, and after all the author's details
  respectively.
\end{keylist}

\subsubsection{Supervisor}
\label{subsubsec:supervisor}

\begin{fontkeylist}{
    /thesis/supervisor/description,
    /thesis/supervisor/first name,
    /thesis/supervisor/last name}
  Specifies the font for the supervisor description, and the supervisor(s)'s
  first and last name.
\end{fontkeylist}

\begin{key}{/thesis/supervisor/description/text (initially Under the supervision of)}
  The text introducing the supervisor(s).
\end{key}

\begin{key}{/thesis/supervisor/separator (initially \string\space)}
  The separator between each supervisor's first and last name.
\end{key}

\begin{keylist}{
    /thesis/supervisor/description/skip (initially 0.2em),
    /thesis/supervisor/inter skip (initially 0pt),
    /thesis/supervisor/skip (initially 3em)}
  The vertical skip between the description and the first supervisor's name,
  between each supervisor, and between the last supervisor and after the last supervisor.
\end{keylist}

\subsubsection{Degree}
\label{subsubsec:degree}

\begin{fontkeylist}{
    /thesis/degree/description,
    /thesis/degree/name}
  Specifies the font for the degree description and the degree name.
\end{fontkeylist}

\begin{key}{/thesis/degree/description/text (initially Submitted in the fulfillment of)}
  The text introducing the degree.
\end{key}

\begin{keylist}{
    /thesis/degree/description/skip (initially 0.2em),
    /thesis/degree/skip (initially 0em plus 1 fill)}
  The vertical skip between the degree description
\end{keylist}

\subsubsection{Date}
\label{subsubsec:date}

\begin{fontkey}{/thesis/title/date}
  Adjust the font for the date.
\end{fontkey}

\subsection{Lettrines}
\label{subsec:lettrines}

\begin{command}{\thesislettrine\marg{letter}\marg{text}}
  Inserts a \href{https://en.wikipedia.org/wiki/Initial}{lettrine} (or
  \emph{initial} or \emph{drop cap}) at the start of a paragraph, and formats
  the remaining text specified in \marg{text} using small caps.  Conventionally,
  it is either the rest of the first word or noun phrase that is included the
  \marg{text}.

  \hfill\begin{minipage}{0.9\textwidth}
    % For some reason, the lettrine doesn't display nicely from the codeexample.
    \thesislettrine{A}{simple example of a lettrine} in use.  \lipsum[2]
  \end{minipage}

\begin{codeexample}[execute code=false]
\thesislettrine{A}{simple example of a lettrine} in use.  \lipsum[2]
\end{codeexample}

  If \LuaLaTeX{} is available, the lettrine will make use of the EB Garamond
  lettrine glyphs.  EB Garamond provides two separate fonts for the foreground
  and background of the lettrines; alternative fonts can be used by adjusting
  |\lettrinebg| and |\lettrinefg| (see below).  If \LuaLaTeX{} is not available,
  the |\thesislettrine| will simply insert a large first character and use the
  primary colour.

  \hfill\begin{minipage}{0.9\textwidth}
    % For some reason, the lettrine doesn't display nicely from the codeexample.
    \setcounter{DefaultLines}{3}
    \renewcommand{\DefaultLoversize}{0}
    \renewcommand{\DefaultLraise}{0}
    \setlength{\DefaultNindent}{0.5ex}
    \lettrine{\textcolor{primary}{A}}{simple example of a lettrine} in use.  \lipsum[2]
  \end{minipage}

\begin{codeexample}[execute code=false]
\thesislettrine{A}{simple example of a lettrine} in use.  \lipsum[2]
\end{codeexample}

  Internally, |\thesislettrine| makes use of the |\lettrine| command from the
  \href{http://ctan.org/pkg/lettrine}{|lettrine|} package which provides various
  other commands to customize the appearance of the lettrines.  In using
  \LuaLaTeX{}, the foreground and background fonts can be adjusted using the
  following two commands:

\end{command}

% Local Variables:
% TeX-master: "../thesis.tex"
% End:
\chapter{Lorem}

\lipsum[1-3]

\section{Quiset Habitant}

\lipsum[4-10]

\subsection{Nullam Cursus}

\lipsum[11-13]

\subsection{Donec et Mi}

\lipsum[14-17]

\section{Dignissim Rutrum}

\lipsum[20-26]

\subsection{Fringilla}

\lipsum[26-28]

\subsection{Aliquet}

\lipsum[29-30]

% Local Variables:
% TeX-master: "../thesis.tex"
% End:
\chapter{Ipsum}

\lipsum[1-3]

\section{Quiset Habitant}

\lipsum[4-10]

\subsection{Nullam Cursus}

\lipsum[11-13]

\subsection{Donec et Mi}

\lipsum[14-17]

\section{Dignissim Rutrum}

\lipsum[20-26]

\subsection{Fringilla}

\lipsum[26-28]

\subsection{Aliquet}

\lipsum[29-30]

% Local Variables:
% TeX-master: "../thesis.tex"
% End:
\include{chapter/conclusion}
\end{document}
\end{codeexample}


\section{Customizations}
\label{sec:customizations}

This thesis class makes use `keys' and `styles' in the same as a lot of
Ti\emph{k}Z packages.  This allows for many customizations to be done easily and
consistently.  The various keys are defined below along with descriptions of
what each does.  They can be customized using the |\thesisset| command.

\begin{command}{\thesisset\marg{keys}}
  This facilitates the change of thesis keys and styles and by default will
  adjust keys in the |/thesis| family of keys.

\begin{codeexample}[execute code=false]
\thesisset{
  title/logo={
    \includegraphics[width=6cm]{my-university-logo.png}
  },
  title/logo skip=3cm,
}
\end{codeexample}
\end{command}

There are a few cases where customizations are handled by other packages, or
where commands are more appropriate.

\subsection{Font Handling}
\label{subsec:font_handling}

The thesis template makes use of a few font, especially in the title, and in
order to handle all these fonts and switch between them easily, a convenience
functions are introduced.

At the core of all this is the |font| handler.  A key can be designated as a
font with |.is font| and can be subsequently accessed with |\thesisgetfont| or
|\thesisfont|.

\begin{handler}{.is font}
  This defines a particular key to be an actual font.  This creates a new family
  with the given key, and defines the following subkeys with default values.
  The font can then be used in conjunction with |\thesisgetfont| and
  |\thesisfont|.

  Since the handler simply turns the key into a family of keys, it is possible
  to add additional subkeys without affecting the functionality of the font
  key.  For example:

  \begin{codeexample}[]
  \thesisset{
    cormorant/.is font,
    cormorant/font=\fontspec{Cormorant},
    cormorant/color=red!75!blue,
    cormorant/shape=\scshape,
    % The following is NOT part of the font keys
    cormorant/text/.initial={I'm in Cormorant!}
  }
  \thesisgetfont{cormorant}{\thesisget{cormorant/text}}
  \end{codeexample}

  The subkeys defined, along with their respective default values, are:
  \begin{description}[font=\normalfont]
  \item[\marg{key}|/size|] (default |1em|) \newline
    Stores the size (with units) of the font;
  \item[\marg{key}|/font|] (default |\normalfont|) \newline
    Defines the font.  Can be used in conjunction with |fontspec|'s |\fontspec|
    command;
  \item[\marg{key}|/color|] (default |black|) \newline
    Changes the colour;
  \item[\marg{key}|/shape|] (default |\upshape|) \newline
    Adjusts the font shape (i.e. |\itshape|, |\scshape|);
  \item[\marg{key}|/series|] (default |\mdseries|) \newline
    Changes the font series (i.e. |\bfseries|, |\mdseries|).
  \end{description}
\end{handler}

\begin{command}{\thesisgetfont\marg{font key}}
  This fetches a font key (that is, one defined with |.is font|), and applies
  the corresponding font to the current context.  This command is analogous to
  commands such as |\bfseries|, |\scshape|, \dots{} in that they will apply to
  the \emph{entire} context and thus should be used within an environment and/or
  braces |{}| in order to restrict the extent of the command.

\begin{codeexample}[]
\thesisset{
  cormorant/.is font,
  cormorant/font=\fontspec{Cormorant},
  cormorant/color=red!75!blue,
  cormorant/shape=\scshape,
}
{\thesisgetfont{cormorant} Hello, World!} Goodbye
\end{codeexample}
\end{command}

\begin{command}{\thesisfont\marg{font key}\marg{text}}
  This fetches a font key (that is, one defined with |.is font|), and applies
  the corresponding font to the specified text.  This command is analogous to
  commands such as |\textbf|\marg{text}, |\textsc|\marg{text}, \dots{} in that
  they apply only the the specified text.

\begin{codeexample}[]
\thesisset{
  cormorant/.is font,
  cormorant/font=\fontspec{Cormorant},
  cormorant/color=red!25!blue,
  cormorant/shape=\scshape,
}
\thesisfont{cormorant}{Hello, World!} Goodbye
\end{codeexample}
\end{command}


\section{Document Information}
\label{sec:document_information}

Information about authors, supervisors, title, \dots{} should be set in the
preamble using the commands provided below or by setting the appropriate key.
If needed, the values can subsequently be accessed from the key.  In order to be
compatible with the usual way \LaTeX{} handles this information, the associated
|\@...| command is also defined

\begin{keylist}{
    /thesis/author/first name,
    /thesis/author/last name}
  Stores the author's first and last name.

  \begin{command}{\author\marg{first name}\marg{last name}}
    A convenience function to set the author.
  \end{command}
\end{keylist}

\begin{key}{/thesis/author/orcid}
  Stores the author's \textsc{Orcid}.

  \begin{command}{\orcid\marg{orcid}}
    A convenience function to set the \textsc{Orcid}.
  \end{command}
\end{key}

\begin{keylist}{
    /thesis/supervisors/\meta{count}/first name,
    /thesis/supervisors/\meta{count}/last name}
  Stores the supervisor's first and last name, where \meta{count} is an integer
  (starting from 1) allowing for multiple supervisors to be handled.  The
  information should always be set with the |\supervisor| command.

  Internally, the number of supervisors in total is recorded in the counter
  |supervisorcount|.  It is then possible to programmatically access each
  supervisor with:

\begin{codeexample}[]
\foreach \i in {1, ..., \arabic{supervisorcount}} {
  \thesisget{supervisors/\i/first name}
  ~\thesisget{supervisors/\i/last name}
}
\end{codeexample}

  \begin{command}{\supervisor\marg{first name}\marg{last name}}
    A convenience function to input supervisors.  This automatically handles the
    mechanics to keep track of the number of supervisors and is the preferred
    way to set the information.  The order in which the supervisors are
    specified is kept.
  \end{command}
\end{keylist}

\begin{key}{/thesis/logo}
  Stores the code for the logo.  Typically, this will be |\includegraphics{...}|.

  \begin{command}{\logo\marg{code}}
    A convenience function to set the logo.
  \end{command}
\end{key}

\begin{key}{/thesis/degree title}
  Stores the degree title.

  \begin{command}{\degreetitle\marg{text}}
    A convenience function to set the degree title
  \end{command}
\end{key}

\begin{key}{/thesis/university}
  Stores the university name.

  \begin{command}{\university\marg{text}}
    A convenience function to set the university name
  \end{command}
\end{key}

\begin{key}{/thesis/department}
  Stores the university department.

  \begin{command}{\department\marg{text}}
    A convenience function to set the university department.
  \end{command}
\end{key}

\begin{key}{/thesis/subject}
  Stores the subject.  This is used primarily to set the |subject| property in
  the \textsc{pdf}.

  \begin{command}{\subject\marg{text}}
    A convenience function to set the subject.
  \end{command}
\end{key}

\begin{key}{/thesis/keywords}
  Stores a list of keywords relating to the thesis.  This is used primarily to
  set the |keywords| property in the \textsc{pdf}.

  \begin{command}{\keywords\marg{text}}
    A convenience function to set the keywords.
  \end{command}
\end{key}


\section{Formatting}
\label{sec:formatting}

The |thesis| class is based on the |scrreprt| class from the
\href{https://www.ctan.org/pkg/koma-script}{KOMA-Script bundle}.  The |thesis|
customizes a lot of the formatting from the |scrreprt|, but these can always be
adjusted further in the preamble of the thesis.  In particular, the font and
formatting of section titles can be adjust with the |\setkomafont| command.

The formatting of the title can be extensively adjusted using keys provided by
the |thesis| class, as documented in \cref{subsec:title_page}.  There are other
miscellaneous features provided by the font, such as lettrines in
\cref{subsec:lettrines}.

\subsection{Draft Mode}
\label{subsec:draft_mode}

Since a thesis can take a long time to compile, it can be worthwhile to enable
|draft| mode by adding |draft| as an option to the document class:

\begin{codeexample}[execute code=false]
\documentclass[draft]{thesis}
\end{codeexample}

When draft mode is enabled, the |thesis| class will automatically add line
numbers and highlight certain errors with a black box near the error.  More
importantly, certain features which are known to significantly increase
compilation time are disabled.  Some changes include:
\begin{itemize}
\item Most features by \href{http://ctan.org/pkg/microtype}{|microtype|} are
  disabled;
\item Hyperlinks are not generated by
  \href{http://ctan.org/pkg/hyperref}{|hyperref|};
\item Images are replaced by placeholders.
\end{itemize}

When the final version of the document is to be generated, |draft| should be
replaced with |final|:

\begin{codeexample}[execute code=false]
\documentclass[final]{thesis}
\end{codeexample}

This ensures that all the features are enabled.

\subsection{Page Layout}
\label{subsec:page_layout}

The page layout can be adjusted using the |\geometry| command from the extensive
\href{http://ctan.org/pkg/geometry}{|geometry|} package.  Please refer to the
|geometry| documentation for a full list of all of its capabilities.  As a rule
of thumb (and ultimately, it is ultimately a matter of taste), it is advisable
to set the length of each line so that there are no more than 60--90 characters
as the line length is long enough to read fairly fast without being so long that
the eye can't find the start of the next line.  The text width can be adjusted
|textwidth| key which will adjust the margins automatically to fit.

\begin{codeexample}[execute code=false]
\geometry{
  textwidth=13cm,
  vmargin={3cm,3cm},
}
\end{codeexample}

\subsection{Title Page}
\label{subsec:title_page}

The thesis title is generated using the |\maketitle| command and can be
extensively customized using various keys provided by the |thesis| class, as
described below.  The information for the title is obtained from the document
information (see \cref{sec:document_information}).  All keys to customize the
title page fall under |/thesis/title|, and since there are many components, they
are further subcategorized as seen below.

\subsubsection{Logo}
\label{subsubsec:logo}

\begin{key}{/thesis/title/logo/skip (initially 2em)}
  The vertical skip after the logo when the logo is present.
\end{key}

\subsubsection{University Name and Department}
\label{subsubsec:university_logo_name_and_department}

\begin{fontkeylist}{/thesis/title/university/name, /thesis/title/university/department}
  The font for the university name and department respectively.
\end{fontkeylist}

\begin{keylist}{
    /thesis/title/university/name/skip (initially 1em),
    /thesis/title/university/department/skip (initially 5em)}
  The vertical skip after the university name and department respectively.  They
  are only used if corresponding text is not empty.
\end{keylist}

\subsubsection{Title and Subtitle}
\label{subsubsec:title_and_subtitle}

\begin{fontkeylist}{
    /thesis/title/title,
    /thesis/title/subtitle}
  The font for the thesis title and subtitle.
\end{fontkeylist}

\begin{key}{/thesis/title/title/skip (initially 5em)}
  The vertical skip \emph{after both} title and subtitle.
\end{key}

\begin{key}{/thesis/title/subtitle/skip (initially 0.5em)}
  The vertical skip \emph{between} the title and subtitle.
\end{key}

\subsubsection{Author}
\label{subsubsec:author}

\begin{fontkeylist}{
    /thesis/title/author/description,
    /thesis/title/author/first name,
    /thesis/title/author/last name,
    /thesis/title/author/orcid}
  Specifies the font for the author description, the author's first and last
  name, and the \textsc{orcid}.
\end{fontkeylist}

\begin{key}{/thesis/title/author/description/text (initially By)}
  The text introducing the author.
\end{key}

\begin{key}{/thesis/title/author/separator (initially \string\space)}
  The separator between the author's first and last name.
\end{key}

\begin{keylist}{
    /thesis/title/author/description/skip (initially 0.2em),
    /thesis/title/author/orcid/skip (initially -0.2em),
    /thesis/title/author/skip (initially 3em)}
  The vertical skip between the description text and the author's name, between
  the author's name and the \textsc{orcid}, and after all the author's details
  respectively.
\end{keylist}

\subsubsection{Supervisor}
\label{subsubsec:supervisor}

\begin{fontkeylist}{
    /thesis/supervisor/description,
    /thesis/supervisor/first name,
    /thesis/supervisor/last name}
  Specifies the font for the supervisor description, and the supervisor(s)'s
  first and last name.
\end{fontkeylist}

\begin{key}{/thesis/supervisor/description/text (initially Under the supervision of)}
  The text introducing the supervisor(s).
\end{key}

\begin{key}{/thesis/supervisor/separator (initially \string\space)}
  The separator between each supervisor's first and last name.
\end{key}

\begin{keylist}{
    /thesis/supervisor/description/skip (initially 0.2em),
    /thesis/supervisor/inter skip (initially 0pt),
    /thesis/supervisor/skip (initially 3em)}
  The vertical skip between the description and the first supervisor's name,
  between each supervisor, and between the last supervisor and after the last supervisor.
\end{keylist}

\subsubsection{Degree}
\label{subsubsec:degree}

\begin{fontkeylist}{
    /thesis/degree/description,
    /thesis/degree/name}
  Specifies the font for the degree description and the degree name.
\end{fontkeylist}

\begin{key}{/thesis/degree/description/text (initially Submitted in the fulfillment of)}
  The text introducing the degree.
\end{key}

\begin{keylist}{
    /thesis/degree/description/skip (initially 0.2em),
    /thesis/degree/skip (initially 0em plus 1 fill)}
  The vertical skip between the degree description
\end{keylist}

\subsubsection{Date}
\label{subsubsec:date}

\begin{fontkey}{/thesis/title/date}
  Adjust the font for the date.
\end{fontkey}

\subsection{Lettrines}
\label{subsec:lettrines}

\begin{command}{\thesislettrine\marg{letter}\marg{text}}
  Inserts a \href{https://en.wikipedia.org/wiki/Initial}{lettrine} (or
  \emph{initial} or \emph{drop cap}) at the start of a paragraph, and formats
  the remaining text specified in \marg{text} using small caps.  Conventionally,
  it is either the rest of the first word or noun phrase that is included the
  \marg{text}.

  \hfill\begin{minipage}{0.9\textwidth}
    % For some reason, the lettrine doesn't display nicely from the codeexample.
    \thesislettrine{A}{simple example of a lettrine} in use.  \lipsum[2]
  \end{minipage}

\begin{codeexample}[execute code=false]
\thesislettrine{A}{simple example of a lettrine} in use.  \lipsum[2]
\end{codeexample}

  If \LuaLaTeX{} is available, the lettrine will make use of the EB Garamond
  lettrine glyphs.  EB Garamond provides two separate fonts for the foreground
  and background of the lettrines; alternative fonts can be used by adjusting
  |\lettrinebg| and |\lettrinefg| (see below).  If \LuaLaTeX{} is not available,
  the |\thesislettrine| will simply insert a large first character and use the
  primary colour.

  \hfill\begin{minipage}{0.9\textwidth}
    % For some reason, the lettrine doesn't display nicely from the codeexample.
    \setcounter{DefaultLines}{3}
    \renewcommand{\DefaultLoversize}{0}
    \renewcommand{\DefaultLraise}{0}
    \setlength{\DefaultNindent}{0.5ex}
    \lettrine{\textcolor{primary}{A}}{simple example of a lettrine} in use.  \lipsum[2]
  \end{minipage}

\begin{codeexample}[execute code=false]
\thesislettrine{A}{simple example of a lettrine} in use.  \lipsum[2]
\end{codeexample}

  Internally, |\thesislettrine| makes use of the |\lettrine| command from the
  \href{http://ctan.org/pkg/lettrine}{|lettrine|} package which provides various
  other commands to customize the appearance of the lettrines.  In using
  \LuaLaTeX{}, the foreground and background fonts can be adjusted using the
  following two commands:

\end{command}

% Local Variables:
% TeX-master: "../thesis.tex"
% End:
\chapter{Lorem}

\lipsum[1-3]

\section{Quiset Habitant}

\lipsum[4-10]

\subsection{Nullam Cursus}

\lipsum[11-13]

\subsection{Donec et Mi}

\lipsum[14-17]

\section{Dignissim Rutrum}

\lipsum[20-26]

\subsection{Fringilla}

\lipsum[26-28]

\subsection{Aliquet}

\lipsum[29-30]

% Local Variables:
% TeX-master: "../thesis.tex"
% End:
\chapter{Ipsum}

\lipsum[1-3]

\section{Quiset Habitant}

\lipsum[4-10]

\subsection{Nullam Cursus}

\lipsum[11-13]

\subsection{Donec et Mi}

\lipsum[14-17]

\section{Dignissim Rutrum}

\lipsum[20-26]

\subsection{Fringilla}

\lipsum[26-28]

\subsection{Aliquet}

\lipsum[29-30]

% Local Variables:
% TeX-master: "../thesis.tex"
% End:
\include{chapter/conclusion}
\end{document}
\end{codeexample}


\section{Customizations}
\label{sec:customizations}

This thesis class makes use `keys' and `styles' in the same as a lot of
Ti\emph{k}Z packages.  This allows for many customizations to be done easily and
consistently.  The various keys are defined below along with descriptions of
what each does.  They can be customized using the |\thesisset| command.

\begin{command}{\thesisset\marg{keys}}
  This facilitates the change of thesis keys and styles and by default will
  adjust keys in the |/thesis| family of keys.

\begin{codeexample}[execute code=false]
\thesisset{
  title/logo={
    \includegraphics[width=6cm]{my-university-logo.png}
  },
  title/logo skip=3cm,
}
\end{codeexample}
\end{command}

There are a few cases where customizations are handled by other packages, or
where commands are more appropriate.

\subsection{Font Handling}
\label{subsec:font_handling}

The thesis template makes use of a few font, especially in the title, and in
order to handle all these fonts and switch between them easily, a convenience
functions are introduced.

At the core of all this is the |font| handler.  A key can be designated as a
font with |.is font| and can be subsequently accessed with |\thesisgetfont| or
|\thesisfont|.

\begin{handler}{.is font}
  This defines a particular key to be an actual font.  This creates a new family
  with the given key, and defines the following subkeys with default values.
  The font can then be used in conjunction with |\thesisgetfont| and
  |\thesisfont|.

  Since the handler simply turns the key into a family of keys, it is possible
  to add additional subkeys without affecting the functionality of the font
  key.  For example:

  \begin{codeexample}[]
  \thesisset{
    cormorant/.is font,
    cormorant/font=\fontspec{Cormorant},
    cormorant/color=red!75!blue,
    cormorant/shape=\scshape,
    % The following is NOT part of the font keys
    cormorant/text/.initial={I'm in Cormorant!}
  }
  \thesisgetfont{cormorant}{\thesisget{cormorant/text}}
  \end{codeexample}

  The subkeys defined, along with their respective default values, are:
  \begin{description}[font=\normalfont]
  \item[\marg{key}|/size|] (default |1em|) \newline
    Stores the size (with units) of the font;
  \item[\marg{key}|/font|] (default |\normalfont|) \newline
    Defines the font.  Can be used in conjunction with |fontspec|'s |\fontspec|
    command;
  \item[\marg{key}|/color|] (default |black|) \newline
    Changes the colour;
  \item[\marg{key}|/shape|] (default |\upshape|) \newline
    Adjusts the font shape (i.e. |\itshape|, |\scshape|);
  \item[\marg{key}|/series|] (default |\mdseries|) \newline
    Changes the font series (i.e. |\bfseries|, |\mdseries|).
  \end{description}
\end{handler}

\begin{command}{\thesisgetfont\marg{font key}}
  This fetches a font key (that is, one defined with |.is font|), and applies
  the corresponding font to the current context.  This command is analogous to
  commands such as |\bfseries|, |\scshape|, \dots{} in that they will apply to
  the \emph{entire} context and thus should be used within an environment and/or
  braces |{}| in order to restrict the extent of the command.

\begin{codeexample}[]
\thesisset{
  cormorant/.is font,
  cormorant/font=\fontspec{Cormorant},
  cormorant/color=red!75!blue,
  cormorant/shape=\scshape,
}
{\thesisgetfont{cormorant} Hello, World!} Goodbye
\end{codeexample}
\end{command}

\begin{command}{\thesisfont\marg{font key}\marg{text}}
  This fetches a font key (that is, one defined with |.is font|), and applies
  the corresponding font to the specified text.  This command is analogous to
  commands such as |\textbf|\marg{text}, |\textsc|\marg{text}, \dots{} in that
  they apply only the the specified text.

\begin{codeexample}[]
\thesisset{
  cormorant/.is font,
  cormorant/font=\fontspec{Cormorant},
  cormorant/color=red!25!blue,
  cormorant/shape=\scshape,
}
\thesisfont{cormorant}{Hello, World!} Goodbye
\end{codeexample}
\end{command}


\section{Document Information}
\label{sec:document_information}

Information about authors, supervisors, title, \dots{} should be set in the
preamble using the commands provided below or by setting the appropriate key.
If needed, the values can subsequently be accessed from the key.  In order to be
compatible with the usual way \LaTeX{} handles this information, the associated
|\@...| command is also defined

\begin{keylist}{
    /thesis/author/first name,
    /thesis/author/last name}
  Stores the author's first and last name.

  \begin{command}{\author\marg{first name}\marg{last name}}
    A convenience function to set the author.
  \end{command}
\end{keylist}

\begin{key}{/thesis/author/orcid}
  Stores the author's \textsc{Orcid}.

  \begin{command}{\orcid\marg{orcid}}
    A convenience function to set the \textsc{Orcid}.
  \end{command}
\end{key}

\begin{keylist}{
    /thesis/supervisors/\meta{count}/first name,
    /thesis/supervisors/\meta{count}/last name}
  Stores the supervisor's first and last name, where \meta{count} is an integer
  (starting from 1) allowing for multiple supervisors to be handled.  The
  information should always be set with the |\supervisor| command.

  Internally, the number of supervisors in total is recorded in the counter
  |supervisorcount|.  It is then possible to programmatically access each
  supervisor with:

\begin{codeexample}[]
\foreach \i in {1, ..., \arabic{supervisorcount}} {
  \thesisget{supervisors/\i/first name}
  ~\thesisget{supervisors/\i/last name}
}
\end{codeexample}

  \begin{command}{\supervisor\marg{first name}\marg{last name}}
    A convenience function to input supervisors.  This automatically handles the
    mechanics to keep track of the number of supervisors and is the preferred
    way to set the information.  The order in which the supervisors are
    specified is kept.
  \end{command}
\end{keylist}

\begin{key}{/thesis/logo}
  Stores the code for the logo.  Typically, this will be |\includegraphics{...}|.

  \begin{command}{\logo\marg{code}}
    A convenience function to set the logo.
  \end{command}
\end{key}

\begin{key}{/thesis/degree title}
  Stores the degree title.

  \begin{command}{\degreetitle\marg{text}}
    A convenience function to set the degree title
  \end{command}
\end{key}

\begin{key}{/thesis/university}
  Stores the university name.

  \begin{command}{\university\marg{text}}
    A convenience function to set the university name
  \end{command}
\end{key}

\begin{key}{/thesis/department}
  Stores the university department.

  \begin{command}{\department\marg{text}}
    A convenience function to set the university department.
  \end{command}
\end{key}

\begin{key}{/thesis/subject}
  Stores the subject.  This is used primarily to set the |subject| property in
  the \textsc{pdf}.

  \begin{command}{\subject\marg{text}}
    A convenience function to set the subject.
  \end{command}
\end{key}

\begin{key}{/thesis/keywords}
  Stores a list of keywords relating to the thesis.  This is used primarily to
  set the |keywords| property in the \textsc{pdf}.

  \begin{command}{\keywords\marg{text}}
    A convenience function to set the keywords.
  \end{command}
\end{key}


\section{Formatting}
\label{sec:formatting}

The |thesis| class is based on the |scrreprt| class from the
\href{https://www.ctan.org/pkg/koma-script}{KOMA-Script bundle}.  The |thesis|
customizes a lot of the formatting from the |scrreprt|, but these can always be
adjusted further in the preamble of the thesis.  In particular, the font and
formatting of section titles can be adjust with the |\setkomafont| command.

The formatting of the title can be extensively adjusted using keys provided by
the |thesis| class, as documented in \cref{subsec:title_page}.  There are other
miscellaneous features provided by the font, such as lettrines in
\cref{subsec:lettrines}.

\subsection{Draft Mode}
\label{subsec:draft_mode}

Since a thesis can take a long time to compile, it can be worthwhile to enable
|draft| mode by adding |draft| as an option to the document class:

\begin{codeexample}[execute code=false]
\documentclass[draft]{thesis}
\end{codeexample}

When draft mode is enabled, the |thesis| class will automatically add line
numbers and highlight certain errors with a black box near the error.  More
importantly, certain features which are known to significantly increase
compilation time are disabled.  Some changes include:
\begin{itemize}
\item Most features by \href{http://ctan.org/pkg/microtype}{|microtype|} are
  disabled;
\item Hyperlinks are not generated by
  \href{http://ctan.org/pkg/hyperref}{|hyperref|};
\item Images are replaced by placeholders.
\end{itemize}

When the final version of the document is to be generated, |draft| should be
replaced with |final|:

\begin{codeexample}[execute code=false]
\documentclass[final]{thesis}
\end{codeexample}

This ensures that all the features are enabled.

\subsection{Page Layout}
\label{subsec:page_layout}

The page layout can be adjusted using the |\geometry| command from the extensive
\href{http://ctan.org/pkg/geometry}{|geometry|} package.  Please refer to the
|geometry| documentation for a full list of all of its capabilities.  As a rule
of thumb (and ultimately, it is ultimately a matter of taste), it is advisable
to set the length of each line so that there are no more than 60--90 characters
as the line length is long enough to read fairly fast without being so long that
the eye can't find the start of the next line.  The text width can be adjusted
|textwidth| key which will adjust the margins automatically to fit.

\begin{codeexample}[execute code=false]
\geometry{
  textwidth=13cm,
  vmargin={3cm,3cm},
}
\end{codeexample}

\subsection{Title Page}
\label{subsec:title_page}

The thesis title is generated using the |\maketitle| command and can be
extensively customized using various keys provided by the |thesis| class, as
described below.  The information for the title is obtained from the document
information (see \cref{sec:document_information}).  All keys to customize the
title page fall under |/thesis/title|, and since there are many components, they
are further subcategorized as seen below.

\subsubsection{Logo}
\label{subsubsec:logo}

\begin{key}{/thesis/title/logo/skip (initially 2em)}
  The vertical skip after the logo when the logo is present.
\end{key}

\subsubsection{University Name and Department}
\label{subsubsec:university_logo_name_and_department}

\begin{fontkeylist}{/thesis/title/university/name, /thesis/title/university/department}
  The font for the university name and department respectively.
\end{fontkeylist}

\begin{keylist}{
    /thesis/title/university/name/skip (initially 1em),
    /thesis/title/university/department/skip (initially 5em)}
  The vertical skip after the university name and department respectively.  They
  are only used if corresponding text is not empty.
\end{keylist}

\subsubsection{Title and Subtitle}
\label{subsubsec:title_and_subtitle}

\begin{fontkeylist}{
    /thesis/title/title,
    /thesis/title/subtitle}
  The font for the thesis title and subtitle.
\end{fontkeylist}

\begin{key}{/thesis/title/title/skip (initially 5em)}
  The vertical skip \emph{after both} title and subtitle.
\end{key}

\begin{key}{/thesis/title/subtitle/skip (initially 0.5em)}
  The vertical skip \emph{between} the title and subtitle.
\end{key}

\subsubsection{Author}
\label{subsubsec:author}

\begin{fontkeylist}{
    /thesis/title/author/description,
    /thesis/title/author/first name,
    /thesis/title/author/last name,
    /thesis/title/author/orcid}
  Specifies the font for the author description, the author's first and last
  name, and the \textsc{orcid}.
\end{fontkeylist}

\begin{key}{/thesis/title/author/description/text (initially By)}
  The text introducing the author.
\end{key}

\begin{key}{/thesis/title/author/separator (initially \string\space)}
  The separator between the author's first and last name.
\end{key}

\begin{keylist}{
    /thesis/title/author/description/skip (initially 0.2em),
    /thesis/title/author/orcid/skip (initially -0.2em),
    /thesis/title/author/skip (initially 3em)}
  The vertical skip between the description text and the author's name, between
  the author's name and the \textsc{orcid}, and after all the author's details
  respectively.
\end{keylist}

\subsubsection{Supervisor}
\label{subsubsec:supervisor}

\begin{fontkeylist}{
    /thesis/supervisor/description,
    /thesis/supervisor/first name,
    /thesis/supervisor/last name}
  Specifies the font for the supervisor description, and the supervisor(s)'s
  first and last name.
\end{fontkeylist}

\begin{key}{/thesis/supervisor/description/text (initially Under the supervision of)}
  The text introducing the supervisor(s).
\end{key}

\begin{key}{/thesis/supervisor/separator (initially \string\space)}
  The separator between each supervisor's first and last name.
\end{key}

\begin{keylist}{
    /thesis/supervisor/description/skip (initially 0.2em),
    /thesis/supervisor/inter skip (initially 0pt),
    /thesis/supervisor/skip (initially 3em)}
  The vertical skip between the description and the first supervisor's name,
  between each supervisor, and between the last supervisor and after the last supervisor.
\end{keylist}

\subsubsection{Degree}
\label{subsubsec:degree}

\begin{fontkeylist}{
    /thesis/degree/description,
    /thesis/degree/name}
  Specifies the font for the degree description and the degree name.
\end{fontkeylist}

\begin{key}{/thesis/degree/description/text (initially Submitted in the fulfillment of)}
  The text introducing the degree.
\end{key}

\begin{keylist}{
    /thesis/degree/description/skip (initially 0.2em),
    /thesis/degree/skip (initially 0em plus 1 fill)}
  The vertical skip between the degree description
\end{keylist}

\subsubsection{Date}
\label{subsubsec:date}

\begin{fontkey}{/thesis/title/date}
  Adjust the font for the date.
\end{fontkey}

\subsection{Lettrines}
\label{subsec:lettrines}

\begin{command}{\thesislettrine\marg{letter}\marg{text}}
  Inserts a \href{https://en.wikipedia.org/wiki/Initial}{lettrine} (or
  \emph{initial} or \emph{drop cap}) at the start of a paragraph, and formats
  the remaining text specified in \marg{text} using small caps.  Conventionally,
  it is either the rest of the first word or noun phrase that is included the
  \marg{text}.

  \hfill\begin{minipage}{0.9\textwidth}
    % For some reason, the lettrine doesn't display nicely from the codeexample.
    \thesislettrine{A}{simple example of a lettrine} in use.  \lipsum[2]
  \end{minipage}

\begin{codeexample}[execute code=false]
\thesislettrine{A}{simple example of a lettrine} in use.  \lipsum[2]
\end{codeexample}

  If \LuaLaTeX{} is available, the lettrine will make use of the EB Garamond
  lettrine glyphs.  EB Garamond provides two separate fonts for the foreground
  and background of the lettrines; alternative fonts can be used by adjusting
  |\lettrinebg| and |\lettrinefg| (see below).  If \LuaLaTeX{} is not available,
  the |\thesislettrine| will simply insert a large first character and use the
  primary colour.

  \hfill\begin{minipage}{0.9\textwidth}
    % For some reason, the lettrine doesn't display nicely from the codeexample.
    \setcounter{DefaultLines}{3}
    \renewcommand{\DefaultLoversize}{0}
    \renewcommand{\DefaultLraise}{0}
    \setlength{\DefaultNindent}{0.5ex}
    \lettrine{\textcolor{primary}{A}}{simple example of a lettrine} in use.  \lipsum[2]
  \end{minipage}

\begin{codeexample}[execute code=false]
\thesislettrine{A}{simple example of a lettrine} in use.  \lipsum[2]
\end{codeexample}

  Internally, |\thesislettrine| makes use of the |\lettrine| command from the
  \href{http://ctan.org/pkg/lettrine}{|lettrine|} package which provides various
  other commands to customize the appearance of the lettrines.  In using
  \LuaLaTeX{}, the foreground and background fonts can be adjusted using the
  following two commands:

\end{command}

% Local Variables:
% TeX-master: "../thesis.tex"
% End: