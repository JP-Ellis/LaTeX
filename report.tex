%%%%%%%%%%%%%%%%%%%%%%%%%%%%%%%%%%%%%%%%%%%%%%%%%%%%%%%%%%%%%%%%%%%%%%%%%%%%%%%%
%%
%% Base Template
%% Copyright (C) 2014  Joshua Ellis
%%
%% This is the template that I use for most of my documents.
%%
%%
%% This LaTeX file is free: you can redistribute it and/or modify it under the
%% terms of the GNU General Public License as published by the Free Software
%% Foundation, either version 3 of the License, or (at your option) any later
%% version.
%%
%% This is distributed in the hope that it will be useful, but WITHOUT ANY
%% WARRANTY; without even the implied warranty of MERCHANTABILITY or FITNESS FOR
%% A PARTICULAR PURPOSE.  See the GNU General Public License for more details.
%%
%%%%%%%%%%%%%%%%%%%%%%%%%%%%%%%%%%%%%%%%%%%%%%%%%%%%%%%%%%%%%%%%%%%%%%%%%%%%%%%%

%%%%%%%%%%%%%%%%%%%%%%%%%%%%%%%%%%%%%%%%%%%%%%%%%%%%%%%%%%%%%%%%%%%%%%%%%%%%%%%%
%% HEADER
%%%%%%%%%%%%%%%%%%%%%%%%%%%%%%%%%%%%%%%%%%%%%%%%%%%%%%%%%%%%%%%%%%%%%%%%%%%%%%%%

%% All LaTeX documents begin with the the `\documentclass` command which serves
%% to set the general style of the document.  It defines the page layout, the
%% chapter/section styles, how figures appear, etc. and may provide some
%% additional environments too.  LaTeX comes with four main classes: `article`,
%% `report`, `book` and `letter`.
%%
%% The [koma-script](https://www.ctan.org/pkg/koma-script) package provides
%% alternatives to replacements for the default classes (`scrartcl`, `scrreprt`,
%% `scrbook`, `scrlttr2`) and provides a lot of options to more easily customize the
%% layout of the page.

%% In general, LaTeX commands start with `\` and consist of letters only.
%% Optional parameters are specified in brackets `[]`, whilst compulsory
%% arguments (if any) are specified in braces `{}`.
\documentclass[twoside, onecolumn, bibliography=totoc, parskip=half*]{scrartcl}
%% Some common optional arguments to the document class are listed below.  The
%% defaults depend on the class use (so see the Koma scripts for bibliography
%% and parskip).
%%
%% - 10pt, 11pt, 12pt
%%   Sets the font size
%%
%% - a4paper, letterpaper, a5paper, ...
%%   Sets the page size, though the `geometry` package allows these options to be
%%   fine-tuned.
%%
%% - onecolumn, twocolumn
%%   Use a one- or two-column layout in the document
%%
%% - oneside, twoside
%%   Formats the pages to be printed either one-sided or two-sided
%%
%% - portait, landscape
%%   Defines the orientation of the page
%%
%% - draft, final
%%   Draft modes highlights some errors in the output, and does not include
%%   pictures (which can save a lot of time in a document with many pictures).
%%
%%
%% The part of the document starting from here through to the
%% `\begin{document}` command is called the preamble.  This is where you include
%% additional packages which provide additional functionalities.  Below are many
%% packages which I personally use at some point or another.
%%
%% More information about nearly every package can be found at
%% ctan.org/pkg/<pkg-name>.

%% Formatting
%%%%%%%%%%%%%%%%%%%%%%%%%%%%%%%%%%%%%%%%%%%%%%%%%%%%%%%%%%%%%%%%%%%%%%%%%%%%%%%%

\usepackage[vmargin={3cm,3cm}]{geometry}  % Customize text width, page height, margins, etc.
% \usepackage{multicol}  % {multicols}{n} environment
% \usepackage{pdflscape} % {landscape} environment

% \usepackage{setspace}  % Line spacing
% \singlespacing
% \onehalfspacing
% \doublespacing

% See http://tex.stackexchange.com/questions/664
% \usepackage[T1]{fontenc} % Do not use with LuaLaTeX
% \usepackage{lmodern}   % Slight improvement to Computer Modern
\usepackage{ebgaramond}  % A nice serif font
% \usepackage{fontspec}  % Allows other fonts to be specified
% \setmainfont{EBGaramond12}
\usepackage{microtype}   % Fine small typographical details

\setkomafont{section}{\normalfont\Huge}
\setkomafont{subsection}{\normalfont\huge}
\setkomafont{subsubsection}{\normalfont\Large}

\setkomafont{sectionentry}{\scshape}
\setkomafont{paragraph}{\normalfont\large\scshape}

%% Language
%%%%%%%%%%%%%%%%%%%%%%%%%%%%%%%%%%%%%%%%%%%%%%%%%%%%%%%%%%%%%%%%%%%%%%%%%%%%%%%%

\usepackage[UKenglish]{babel} % Set up the language
\usepackage{csquotes}
\usepackage{siunitx}          % Provides \SI{1}{\metre}, also \ang{90}.

%% Graphics & Figure
%%%%%%%%%%%%%%%%%%%%%%%%%%%%%%%%%%%%%%%%%%%%%%%%%%%%%%%%%%%%%%%%%%%%%%%%%%%%%%%%

\usepackage{graphicx}   % Allow graphics to be included
\usepackage{xcolor}     % Define and use colours
% \usepackage{subcaption} % Subfigures inside a figure

% Keep all pictures in the './img/' sub-directory.
\graphicspath{{./img/}}

% \usepackage{tikz}       % Powerful drawing language
% \usepackage{pgfplots}   % Plotting with LaTeX

%% TikZ pictures and plots can significantly increase the time it takes to
%% produce the output.  The `external` TikZ library library defers the creation
%% of these figures to a sub-process which creates a separate PDF file which is
%% then simply imported into the main document.  To call the sub-process, you
%% have to execute the appropriate makefile.  If you are using LatexMk, you can
%% use the `.latexmkrc` to automatically do this for you.
%%
%% The following setup works on Linux, and should work on OS X too.
% \usetikzlibrary{external}
% \tikzexternalize[shell escape=-shell-escape, prefix=pgf-img/]
% \immediate\write18{mkdir -p pgf-img}
% \tikzset{
%     external/mode=list and make,
%     external/system call={
%         lualatex \tikzexternalcheckshellescape -halt-on-error -interaction=batchmode -jobname="\image" "\texsource" || rm "\image.pdf"},
% }

%% Math Packages
%%%%%%%%%%%%%%%%%%%%%%%%%%%%%%%%%%%%%%%%%%%%%%%%%%%%%%%%%%%%%%%%%%%%%%%%%%%%%%%%

\usepackage{amsmath}   % The core math package
\usepackage{amssymb}   % Defines additional math fonts 
\usepackage{mathtools} % Various extra maths functions
% \usepackage{cancel}    % Show cancellations with \cancel{}
% \usepackage{autonum}   % Only number referenced equations (must be loaded after cleverref)

\usepackage[cmintegrals,varg]{newtxmath} % Nice math with Garamond

%% Define \withnumber which forces the line to have number
\newcommand{\withnumber}{\refstepcounter{equation}\tag{\theequation}}

%% Allows page breaks in math (1 = avoid if possible, 4 = whenever)
%% Page breaks can be avoided at particular places by using \\*
\allowdisplaybreaks[2]

%% Tables
%%%%%%%%%%%%%%%%%%%%%%%%%%%%%%%%%%%%%%%%%%%%%%%%%%%%%%%%%%%%%%%%%%%%%%%%%%%%%%%%

\usepackage{array}     % New column types, including >{}x<{}
\usepackage{booktabs}  % Provides nicer horizontal lines
\usepackage{multirow}  % Allows cells to span multiple rows
%\usepackage{longtable} % Allows for tables to span multiple pages

%% Define the maths version of clr columns.
\newcolumntype{C}{>{\(}c<{\)}}
\newcolumntype{L}{>{\(}l<{\)}}
\newcolumntype{R}{>{\(}r<{\)}}

%% Other Packages
%%%%%%%%%%%%%%%%%%%%%%%%%%%%%%%%%%%%%%%%%%%%%%%%%%%%%%%%%%%%%%%%%%%%%%%%%%%%%%%%

% \usepackage{enumitem}                  % Easily customize lists
\usepackage[pagecontinue=false]{pageslts} % Reference the last page
\usepackage{hyperref}                    % Automatically inserts hyperlinks.
\usepackage{cleveref}                    % Use `\cref{}` to reference anything

\usepackage{lipsum}                      % For placeholder text

%% Glossary
%%%%%%%%%%%%%%%%%%%%%%%%%%%%%%%%%%%%%%%%%%%%%%%%%%%%%%%%%%%%%%%%%%%%%%%%%%%%%%%%
%% This package requires `makeglossaries` to be run after the initial run of LaTeX
%% so that the glossary is generated and the a second run of LaTeX is need to
%% included the newly generated glossary.

% hyperref should be loaded first
\usepackage[toc]{glossaries}
\renewcommand*{\glstextformat}[1]{\textsf{#1}}
\setglossarystyle{index}

\loadglsentries{glossary}

\makeglossaries

%% Bibliography
%%%%%%%%%%%%%%%%%%%%%%%%%%%%%%%%%%%%%%%%%%%%%%%%%%%%%%%%%%%%%%%%%%%%%%%%%%%%%%%%

%% hyperref should be loaded first
\usepackage[
    backend=biber,
    autocite=inline,
    style=phys,
    biblabel=brackets,
]{biblatex}

\addbibresource{references.bib}

%% Other modifications
%%%%%%%%%%%%%%%%%%%%%%%%%%%%%%%%%%%%%%%%%%%%%%%%%%%%%%%%%%%%%%%%%%%%%%%%%%%%%%%%

%% This is not a CTAN package but instead contains a whole lot of small scripts I
%% use nearly all the time.
\usepackage[theorems=false]{jpellis}

%% Modify the skip after each paragraph
\setlength{\parindent}{1em}

%% Define some slightly nicer colors 
\definecolor{link-color}{RGB}{96 0 0}
\definecolor{cite-color}{RGB}{0 96 0}
\definecolor{file-color}{RGB}{0 0 96}
\definecolor{url-color}{RGB}{0 0 96}
\definecolor{link-border-color}{RGB}{255 159 159}
\definecolor{cite-border-color}{RGB}{159 255 159}
\definecolor{file-border-color}{RGB}{159 159 255}
\definecolor{url-border-color}{RGB}{159 159 255}

\hypersetup{
  %% When `colorlinks` is true, all links will be coloured which looks nice in
  %% digital version of the document but not in print.  If the document is
  %% intended for printing, then `colorlinks` should set to false.
  colorlinks=true,      
  linkcolor=link-color,
  citecolor=cite-color,
  filecolor=file-color,
  urlcolor=url-color,
  linkbordercolor=link-border-color,
  citebordercolor=cite-border-color,
  urlbordercolor=url-border-color,
}

%% Document Information
%%%%%%%%%%%%%%%%%%%%%%%%%%%%%%%%%%%%%%%%%%%%%%%%%%%%%%%%%%%%%%%%%%%%%%%%%%%%%%%%

%% Define a few shorthand commands.  The `makeatletter' and `makeatother' allows
%% the use of '@' in commands which is reserved for hidden functions.
\makeatletter
\newcommand{\@degreetitle}{}
\newcommand{\degreetitle}[1]{\renewcommand{\@degreetitle}{#1}}

\newcommand{\@department}{}
\newcommand{\department}[1]{\renewcommand{\@department}{#1}}

\newcommand{\@university}{}
\newcommand{\university}[1]{\renewcommand{\@university}{#1}}

\newcommand{\@keywords}{}
\newcommand{\keywords}[1]{\renewcommand{\@keywords}{#1}}

\AtBeginDocument{
  \hypersetup{
    pdftitle={\@title},
    pdfauthor={\@author},
    pdfsubject={\@subject},
    pdfkeywords={\@keywords},
  }
}

\makeatother

\title{Writing in \texorpdfstring{\LaTeX}{LaTeX}}
\subtitle{A Practical Introduction to \texorpdfstring{\LaTeX}{LaTeX}}

\subject{LaTeX}
\keywords{}

\author{Joshua \scshape Ellis}
\degreetitle{Masters of Science (Physics)}
\department{School of Physics}
\university{The University of Melbourne}

\date{\today}


%%%%%%%%%%%%%%%%%%%%%%%%%%%%%%%%%%%%%%%%%%%%%%%%%%%%%%%%%%%%%%%%%%%%%%%%%%%%%%%%
%% DOCUMENT
%%%%%%%%%%%%%%%%%%%%%%%%%%%%%%%%%%%%%%%%%%%%%%%%%%%%%%%%%%%%%%%%%%%%%%%%%%%%%%%%
\begin{document}
\pagenumbering{arabic}

%% Title
%%%%%%%%%%%%%%%%%%%%%%%%%%%%%%%%%%%%%%%%%%%%%%%%%%%%%%%%%%%%%%%%%%%%%%%%%%%%%%%%
\pagestyle{empty}

\begin{titlepage}
  \makeatletter

  \begin{center}
    \vspace*{2.5cm}

    \Huge \textbf{\@title} \\[0em]
    \rule{\linewidth}{2pt}
    \huge \textsc{\@subtitle}\\[6em]

    \large By \\[1cm]
    \huge \@author \\[0.5ex]
    \Large \normalfont \@degreetitle

    \vfill

    \Large \@department \\[1ex]
    \Large \@university

    \vfill

    \large \@date
  \end{center}

\makeatother
\end{titlepage}

%% Table of Content
%%%%%%%%%%%%%%%%%%%%%%%%%%%%%%%%%%%%%%%%%%%%%%%%%%%%%%%%%%%%%%%%%%%%%%%%%%%%%%%%
\cleardoublepage
\pagestyle{plain}
\pagenumbering{roman}

\tableofcontents

%%%%%%%%%%%%%%%%%%%%%%%%%%%%%%%%%%%%%%%%%%%%%%%%%%%%%%%%%%%%%%%%%%%%%%%%%%%%%%%%
%% CONTENT
%%%%%%%%%%%%%%%%%%%%%%%%%%%%%%%%%%%%%%%%%%%%%%%%%%%%%%%%%%%%%%%%%%%%%%%%%%%%%%%%
\cleardoublepage
\pagestyle{headings}
\pagenumbering{arabic}

\section{Introduction}

\Gls{latex} is a program designed to typeset documents.  Unlike programs such as
\href{http://office.microsoft.com/en-au/word/}{Microsoft Word} and
\href{http://www.libreoffice.org/discover/writer/}{LibreOffice Writer}, LaTeX
makes an effort to ensure that everything is placed correctly with the
appropriate spacing.

\subsection{Mathematics}
\label{subsec:mathematics}

\Gls{latex} also handles mathematics very nicely; for example, I can write
Einstein's most famous equation inline: \(E = mc^2\).  Inline maths in \LaTeX~is
scaled appropriately so that it fits the line as can be demonstrated with
Newton's second law: \(\vt F = \ddfrac{\vt p}{t}\).  Newton's second law could
also be written at \(\vt F = \dd \vt p / \dd t\), where no scaling is required.

Equations can also be placed on the own line within a text if they are a little
cumbersome to write inline, such the Schr\"odinger equation:
\begin{equation}
    \label{eq:schrodinger_equation}
    i \hbar \pfrac{\Psi(\vt x, t)}{t} = - \frac{\hbar^2}{2m} \nabla^2 \Psi(\vt x, t) + V(\vt x, t) \Psi(\vt x, t).
\end{equation}
When there are multiple equations, these can be aligned nicely to make it
clearer.  The location which is aligned is done at the symbol after the
\verb|&|; typically, this is done at the equal sign as shown with Maxwell's
equation:
\begin{subequations}
    \label{eq:Maxwell_equations}
    \begin{align}
        \label{eq:Gauss_law}\withnumber
        \nabla \cdot \vt E &= \frac{\rho}{\varepsilon_0} \\
        \label{eq:Gauss_law_magnetism}\withnumber
        \nabla \cdot \vt B &= 0 \\
        \label{eq:Maxwell-Faraday_equation}\withnumber
        \nabla \times \vt E &= - \pfrac{\vt B}{t} \\
        \label{eq:Ampere_law}\withnumber
        \nabla \times \vt B &= \mu_0 \vt J + \mu_0 \varepsilon_0 \pfrac{\vt E}{t}
    \end{align}
\end{subequations}
Alternatively, these can all be much more succinctly written as
\begin{equation}
  -\frac{1}{4} F^{\mu\nu} F_{\mu\nu} = 0.
\end{equation}

\LaTeX~also takes care of all the referencing with \verb|\label{}| and
\verb|\ref{}|.  Schr\"odinger's equation for example is equation
\eqref{eq:schrodinger_equation}, whilst Maxwell's equations are listed under
\eqref{eq:Maxwell_equations} and Amp\`ere's law in particular is
\eqref{eq:Ampere_law}.

\subsection{Glossary and Bibliography}
\label{subsec:glossary_and_bibliography}

\Gls{latex}'s referencing capabailities do not end referencing particular equation
(c.f.~\S\ref{subsec:mathematics}) but can also be used to refer to particular
sections.  Additionally with the aid of external programs, in depth
cross-referencing of terms in order to produce a \gls{glossary} is possible.

More commonly, \LaTeX~is used in combination with \gls{bibtex} and \gls{biber} to
produce bibliographies.  This automatically takes care of inserting the proper
reference to some article \cite{Smith2013}, and I can easily cite the author and
year of the article: \citeauthor{Smith2013}, \citeyear{Smith2013}.


%% Just some filler text now...
\clearpage
\section{Lorem Ipsum}

\lipsum[1-4]

\subsection{Aliquam Dolor}

\lipsum[6-9]

\subsection{Dius Non Odo}

\lipsum[10]

\subsubsection{Dignissim Dui}

\lipsum[11-12]

\subsubsection{Laoreet Vitae}

\lipsum[13-15]

\paragraph{Egestas Est} \lipsum[16]

\paragraph{Nam Commodo} \lipsum[17]

\lipsum[18-22]


%%%%%%%%%%%%%%%%%%%%%%%%%%%%%%%%%%%%%%%%%%%%%%%%%%%%%%%%%%%%%%%%%%%%%%%%%%%%%%%%
%% APPENDIX
%%%%%%%%%%%%%%%%%%%%%%%%%%%%%%%%%%%%%%%%%%%%%%%%%%%%%%%%%%%%%%%%%%%%%%%%%%%%%%%%
\cleardoublepage
\appendix
\pagenumbering{Roman}
\pagestyle{plain}

%% Glossary
%%%%%%%%%%%%%%%%%%%%%%%%%%%%%%%%%%%%%%%%%%%%%%%%%%%%%%%%%%%%%%%%%%%%%%%%%%%%%%%%
\cleardoublepage

\printglossaries

%% Bibliography
%%%%%%%%%%%%%%%%%%%%%%%%%%%%%%%%%%%%%%%%%%%%%%%%%%%%%%%%%%%%%%%%%%%%%%%%%%%%%%%%
\cleardoublepage
\pagestyle{plain}
\printbibliography

\end{document}

%%% Local Variables:
%%% TeX-master: t
%%% End:
