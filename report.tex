%%%%%%%%%%%%%%%%%%%%%%%%%%%%%%%%%%%%%%%%%%%%%%%%%%%%%%%%%%%%%%%%%%%%%%%%%%%%%%%%
%
% Report Template
% Copyright (C) 2014  Joshua Ellis
%
% This is the template that I use for most of my documents.
%
%
% This LaTeX file is free: you can redistribute it and/or modify it under the
% terms of the GNU General Public License as published by the Free Software
% Foundation, either version 3 of the License, or (at your option) any later
% version.
%
% This is distributed in the hope that it will be useful, but WITHOUT ANY
% WARRANTY; without even the implied warranty of MERCHANTABILITY or FITNESS FOR
% A PARTICULAR PURPOSE.  See the GNU General Public License for more details.
%
%%%%%%%%%%%%%%%%%%%%%%%%%%%%%%%%%%%%%%%%%%%%%%%%%%%%%%%%%%%%%%%%%%%%%%%%%%%%%%%%

%%%%%%%%%%%%%%%%%%%%%%%%%%%%%%%%%%%%%%%%%%%%%%%%%%%%%%%%%%%%%%%%%%%%%%%%%%%%%%%%
%% HEADER
%%%%%%%%%%%%%%%%%%%%%%%%%%%%%%%%%%%%%%%%%%%%%%%%%%%%%%%%%%%%%%%%%%%%%%%%%%%%%%%%

\documentclass[a4paper,twoside,11pt,final]{article}
% The options are listed below, with the default (for the article class) marked
% by '*'.
%
% - '10pt', 11pt, 12pt
%   Sets the font size of the document
%
% - 'letterpaper', executivepaper, legalpaper, a4paper, a5paper, b5paper
%   Sets the page size
%
% - twocolumn
%   Two columns will be used through the document
%
% - 'oneside', twoside
%   Formats the pages to be printed either one-sided or two-sided
%
% - landscape
%   Use landscape mode
%
% - draft, 'final'
%   With draft enabled, errors are indicated in the output and images are
%   replaced by empty frames.

%% Formatting
%%%%%%%%%%%%%%%%%%%%%%%%%%%%%%%%%%%%%%%%%%%%%%%%%%%%%%%%%%%%%%%%%%%%%%%%%%%%%%%%

\usepackage{geometry} % Easily change margin sizes (and many more dimensions)
\usepackage{fancyhdr} % Nice headers
\usepackage{multicol} % Provides {multicols}{n} for a n-column section
\usepackage{pdflscape} % Provides {landscape} environment

%\usepackage{setspace} % Line spacing
%\singlespacing        % 1-spacing (default)
%\onehalfspacing       % 1,5-spacing
%\doublespacing        % 2-spacing

%% Language
%%%%%%%%%%%%%%%%%%%%%%%%%%%%%%%%%%%%%%%%%%%%%%%%%%%%%%%%%%%%%%%%%%%%%%%%%%%%%%%%

\usepackage[UKenglish]{babel}
\usepackage[T1]{fontenc}
\usepackage{siunitx}
% siunitx should be used to typeset all numbers with units.  It ensures that the
% correct symbol is used with the correct spacing.  It also provides \ang{} to
% specify angles (in degrees).

%% Graphics & Figure
%%%%%%%%%%%%%%%%%%%%%%%%%%%%%%%%%%%%%%%%%%%%%%%%%%%%%%%%%%%%%%%%%%%%%%%%%%%%%%%%

\usepackage[pdftex]{graphicx}   % For loading graphic files
\usepackage{subcaption}         % Subfigures inside a figure
%\usepackage{wrapfig}            % Small figures on side of page (use sparingly)

\usepackage{tikz,pgfplots}
\pgfplotsset{compat=1.10}

% To keep the folder tidy, place all the images in the './img/' folder
\graphicspath{{./img/}}

% Set up PGF externalization
% Note that this requires the folder pgf-img to already exist
\usetikzlibrary{external}
\tikzexternalize[shell escape=-shell-escape, prefix=pgf-img/]
\tikzset{
    external/mode=list and make,
    external/system call={
        lualatex \tikzexternalcheckshellescape -halt-on-error -interaction=batchmode -jobname="\image" "\texsource" || rm "\image.pdf"},
}

% Place customizations for TikZ and PGFplots here
% \tikzset{
%
% }

% \pgfplotsset{
%
% }

%% Math Packages
%%%%%%%%%%%%%%%%%%%%%%%%%%%%%%%%%%%%%%%%%%%%%%%%%%%%%%%%%%%%%%%%%%%%%%%%%%%%%%%%

\usepackage{amsmath,amsfonts,amssymb} % The core packages for math
\usepackage{mathtools}                % Extra features
%\usepackage{cancel}                   % To show cancellations with \cancel{}

% Add tags to referenced lines only
\mathtoolsset{showonlyrefs,showmanualtags}
% Define \withnumber which forces a single line to have a number
\def\withnumber{\refstepcounter{equation}\tag{\theequation}}

% Allows page breaks in math (1 = avoid if possible, 4 = whenever)
% Page breaks can be avoided at particular places by using \\*
\allowdisplaybreaks[2]

%% Tables
%%%%%%%%%%%%%%%%%%%%%%%%%%%%%%%%%%%%%%%%%%%%%%%%%%%%%%%%%%%%%%%%%%%%%%%%%%%%%%%%

\usepackage{array}     % New column types, including >{}x<{}
\usepackage{booktabs}  % Provides nicer horizontal lines
\usepackage{multirow}  % Allows cells to span multiple rows
%\usepackage{longtable} % Allows for tables to span multiple pages

% Define the maths version of clr columns.
\newcolumntype{C}{>{\(}c<{\)}}
\newcolumntype{L}{>{\(}l<{\)}}
\newcolumntype{R}{>{\(}r<{\)}}

%% Other Packages
%%%%%%%%%%%%%%%%%%%%%%%%%%%%%%%%%%%%%%%%%%%%%%%%%%%%%%%%%%%%%%%%%%%%%%%%%%%%%%%%

%\usepackage[version=3]{mhchem} % For chemistry-related stuff
\usepackage{enumitem}    % Can customize {enumerate} and {itemize} lists
%\usepackage{algorithm2e} % To write up algorithms
\usepackage{lastpage}    % Allows referencing the last page
\usepackage{hyperref}    % Automatically inserts hyperlinks.
%\usepackage{cleveref}    % A very good way to manage references, but
                         % incompatible with mathtools' showonlyrefs and
                         % showmanualtags.
\usepackage{lipsum}      % Allows for placeholder text

%% Glossary
%%%%%%%%%%%%%%%%%%%%%%%%%%%%%%%%%%%%%%%%%%%%%%%%%%%%%%%%%%%%%%%%%%%%%%%%%%%%%%%%
% This package requires makeglossaries to be run after the initial run of LaTeX
% so that the glossary is generated and the a second run of LaTeX is need to
% included the newly generated glossary.

% hyperref should be loaded first
\usepackage[toc]{glossaries}
\renewcommand*{\glstextformat}[1]{\textsf{#1}}
\setglossarystyle{index}

\loadglsentries{glossary}

\makeglossaries

%% Bibliography
%%%%%%%%%%%%%%%%%%%%%%%%%%%%%%%%%%%%%%%%%%%%%%%%%%%%%%%%%%%%%%%%%%%%%%%%%%%%%%%%

% hyperref should be loaded first
\usepackage[
    backend=biber,
    autocite=inline,
    style=phys,
    biblabel=brackets,
]{biblatex}

\addbibresource{references.bib}

%% Other modifications
%%%%%%%%%%%%%%%%%%%%%%%%%%%%%%%%%%%%%%%%%%%%%%%%%%%%%%%%%%%%%%%%%%%%%%%%%%%%%%%%

\usepackage{jpellis}

% Modify the skip after each paragraph
\setlength{\parskip}{1ex plus 0.5ex minus 0.2ex}
\setlength{\headheight}{25.23pt}

\definecolor{link-color}{RGB}{0 0 75}
\definecolor{cite-color}{RGB}{75 0 25}
\definecolor{file-color}{RGB}{75 25 0}
\definecolor{url-color}{RGB}{75 25 0}
\definecolor{link-border-color}{RGB}{100 200 255}
\definecolor{cite-border-color}{RGB}{255 0 150}
\definecolor{url-border-color}{RGB}{255 150 0}

\hypersetup{
    colorlinks=true,      % If colorlinks is false, a border is drawn instead
                          % which does not appear in print.  If it is true, the
                          % font is coloured and does appear in print.
    linkcolor=link-color,
    citecolor=cite-color,
    filecolor=file-color,
    urlcolor=url-color,
    linkbordercolor=link-border-color,
    citebordercolor=cite-border-color,
    urlbordercolor=url-border-color,
}

%%%%%%%%%%%%%%%%%%%%%%%%%%%%%%%%%%%%%%%%%%%%%%%%%%%%%%%%%%%%%%%%%%%%%%%%%%%%%%%%
%% DOCUMENT
%%%%%%%%%%%%%%%%%%%%%%%%%%%%%%%%%%%%%%%%%%%%%%%%%%%%%%%%%%%%%%%%%%%%%%%%%%%%%%%%
\begin{document}

%% Title
%%%%%%%%%%%%%%%%%%%%%%%%%%%%%%%%%%%%%%%%%%%%%%%%%%%%%%%%%%%%%%%%%%%%%%%%%%%%%%%%
\pagestyle{empty}

\newcommand{\HRule}{\rule{\linewidth}{0.5mm}}

\begin{titlepage}

    \begin{center}
        \textsc{\huge Lorem Ipsum}\\[3cm]

        \HRule \\[0.5cm]
        \Huge \textbf{Example Document}\\[0.5cm]
        \HRule \\[1.5cm]

        \begin{minipage}{0.4\textwidth}
        \begin{center}

        \large By \\[0.75cm]
        \huge Joshua \scshape Ellis \\[0.5cm]
        \normalsize \normalfont Bachelor of Science \\
        The University of Melbourne \\

        \end{center}
        \end{minipage}

        \vfill

        \large \today
    \end{center}

\newpage

\end{titlepage}

%% Table of Content
%%%%%%%%%%%%%%%%%%%%%%%%%%%%%%%%%%%%%%%%%%%%%%%%%%%%%%%%%%%%%%%%%%%%%%%%%%%%%%%%

\pagestyle{empty}
\tableofcontents
\clearpage


%%%%%%%%%%%%%%%%%%%%%%%%%%%%%%%%%%%%%%%%%%%%%%%%%%%%%%%%%%%%%%%%%%%%%%%%%%%%%%%%
%% CONTENT
%%%%%%%%%%%%%%%%%%%%%%%%%%%%%%%%%%%%%%%%%%%%%%%%%%%%%%%%%%%%%%%%%%%%%%%%%%%%%%%%
\pagestyle{fancy}


\section{Introduction}

\Gls{latex} is a program designed to typeset documents.  Unlike programs such as
\href{http://office.microsoft.com/en-au/word/}{Microsoft Word} and
\href{http://www.libreoffice.org/discover/writer/}{LibreOffice Writer}, LaTeX
makes an effort to ensure that everything is placed correctly with the
appropriate spacing.

\subsection{Mathematics}
\label{subsec:mathematics}

\Gls{latex} also handles mathematics very nicely; for example, I can write
Einstein's most famous equation inline: \(E = mc^2\).  Inline maths in \LaTeX~is
scaled appropriately so that it fits the line as can be demonstrated with
Newton's second law: \(\vt F = \ddfrac{\vt p}{t}\).  Newton's second law could
also be written at \(\vt F = \dd \vt p / \dd t\), where no scaling is required.

Equations can also be placed on the own line within a text if they are a little
cumbersome to write inline, such the Schr\"odinger equation:
\begin{equation}
    \label{eq:schrodinger_equation}
    i \hbar \pfrac{\Psi(\vt x, t)}{t} = - \frac{\hbar^2}{2m} \nabla^2 \Psi(\vt x, t) + V(\vt x, t) \Psi(\vt x, t).
\end{equation}
When there are multiple equations, these can be aligned nicely to make it
clearer.  The location which is aligned is done at the symbol after the
\verb|&|; typically, this is done at the equal sign as shown with Maxwell's
equation:
\begin{subequations}
    \label{eq:Maxwell_equations}
    \begin{align}
        \label{eq:Gauss_law}\withnumber
        \nabla \cdot \vt E &= \frac{\rho}{\varepsilon_0} \\
        \label{eq:Gauss_law_magnetism}\withnumber
        \nabla \cdot \vt B &= 0 \\
        \label{eq:Maxwell-Faraday_equation}\withnumber
        \nabla \times \vt E &= - \pfrac{\vt B}{t} \\
        \label{eq:Ampere_law}\withnumber
        \nabla \times \vt B &= \mu_0 \vt J + \mu_0 \varepsilon_0 \pfrac{\vt E}{t}
    \end{align}
\end{subequations}

\LaTeX~also takes care of all the referencing with \verb|\label{}| and
\verb|\ref{}|.  Schr\"odinger's equation for example is equation
\eqref{eq:schrodinger_equation}, whilst Maxwell's equations are listed under
\eqref{eq:Maxwell_equations} and Amp\`ere's law in particular is
\eqref{eq:Ampere_law}.

\subsection{Glossary and Bibliography}
\label{subsec:glossary_and_bibliography}

\Gls{latex}'s referencing capabailities do not end referencing particular equation
(c.f.~\S\ref{subsec:mathematics}) but can also be used to refer to particular
sections.  Additionally with the aid of external programs, in depth
cross-referencing of terms in order to produce a \gls{glossary} is possible.

More commonly, LaTeX is used in combination with \gls{bibtex} and \gls{biber} to
produce bibliographies.  This automatically takes care of inserting the proper
reference to some article \cite{Smith2013}, and I can easily cite the author and
year of the article: \citeauthor{Smith2013}, \citeyear{Smith2013}.


%% Just some filler text now...
\clearpage
\section{Lorem Ipsum}

\lipsum[1-4]

\subsection{Aliquam Dolor}

\lipsum[6-9]

\subsection{Dius Non Odo}

\lipsum[10]

\subsubsection{Dignissim Dui}

\lipsum[11-12]

\subsubsection{Laoreet Vitae}

\lipsum[13-15]

\paragraph{Egestas Est} \lipsum[16]

\paragraph{Nam Commodo} \lipsum[17]


\lipsum[18-22]





%%%%%%%%%%%%%%%%%%%%%%%%%%%%%%%%%%%%%%%%%%%%%%%%%%%%%%%%%%%%%%%%%%%%%%%%%%%%%%%%
%% GLOSSARY
%%%%%%%%%%%%%%%%%%%%%%%%%%%%%%%%%%%%%%%%%%%%%%%%%%%%%%%%%%%%%%%%%%%%%%%%%%%%%%%%
\clearpage
\pagestyle{empty}

\printglossaries


%%%%%%%%%%%%%%%%%%%%%%%%%%%%%%%%%%%%%%%%%%%%%%%%%%%%%%%%%%%%%%%%%%%%%%%%%%%%%%%%
%% BIBLIOGRAPHY
%%%%%%%%%%%%%%%%%%%%%%%%%%%%%%%%%%%%%%%%%%%%%%%%%%%%%%%%%%%%%%%%%%%%%%%%%%%%%%%%
\clearpage
\pagestyle{empty}

\printbibliography

\end{document}

%%% Local Variables:
%%% TeX-master: t
%%% End:
